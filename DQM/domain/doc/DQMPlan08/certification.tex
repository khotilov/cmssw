
In the certification system detector hardware state information is combined with results 
from DQM quality tests providing certification bits for event selection in physics analyses.
The data certification system is part of the offline DQM processing at Tier-0 
(and Tier-1 for reprocessings). The following proposal for a data certification workflow and persistency has been circulated. This proposal concerns decisions for
\begin{itemize}
\item the retrieval and calculation of certification information (section \ref{sec:cert:calc})
\item the persistent storage of this information (section \ref{sec:cert:schema})
\item necessary steps for the commissioning of a baseline system (section \ref{sec:cert:plan})
\end{itemize}

%Reference to wiki pages
  
\subsection{Retrieval and calculation of certification information}
\label{sec:cert:calc}

Certification algorithms will be run as part of the DQM histogram harvesting and analysis step (2nd step). The certification decisions will be determined per luminosity section based on input from
\begin{itemize}
\item DCS  HV
\item RunInfo DAQ/RCMS
\item DQM online
\item DQM offline.
\end{itemize}
        
The relevant conditions information from DCS, RCMS and DQM online are stored in Orcoff and can be retrieved from there. Appropriate certificiation algorithms to combine these inputs will  be provided by the DPG. 

The granularity should be detector segments, i.e. appropriate fractions of subdetectors, e.g. for Tracker the segmentation could be TIB, TOB, TID, TEC, PXB, PXF (see page 3 of \cite{datacert:gutsche}). For each of the segments a number of inputs will be filled, namely DCS, DAQ, online DQM, offline DQM (and potentially Power and Cooling, as indicated at bottom of page 4 of \cite{datacert:gutsche}).
The certification algorithms yield floating point numbers for each detector segment, representing the working fraction of the respective detector segment.
   
For each detector segment appropriate thresholds are defined such that binary decisions, status bits "good" or "bad", can be obtained from the floating point numbers. In case no number was calculated
the decision bit "unknown" is set.

\subsection{Certification data storage schema}
\label{sec:cert:schema}

The floating point numbers as well as the detector status bits "good/bad/unknown" are stored in the conditions database (Orcoff) and in DBS for  each detector segment.

The schema in DBS and in Orcoff are identical.

The present DBS schema has been explained in \cite{datacert:gutsche}. The global and detailed information to be stored is sketched on page 8  of \cite{datacert:gutsche} for the example of tracker and TIB. 

\subsection{Development and Commissioning}
\label{sec:cert:plan}

As a first step, in order to gain experience, a python-script will be provided, which allows to manually set DQM-online quality global and detailed bits for each detector and detector segment. This script will be ready for use during CR0T and the online-DQM shift person in P5 will be in charge of filling it. The information will be in units of runs, not luminosity sections.

% Further steps
% It will be important to combine the knowledge of the DPG about possible detector failures
% with physics analysis requirements 
% i.e. synergy between POG/PAG and DPG.

%==========================================================
% this subsection is common to all sections, please fill in
\subsection{Data Certification System  Integration and Operation}

\subsubsection*{Integration}
\subsubsection*{Operation}
