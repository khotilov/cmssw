\documentclass[a4paper]{jpconf}
\usepackage[dvipdfm]{graphicx}
\usepackage[dvipdfm]{hyperref}
\usepackage{subfigure}
\usepackage{mediabb}
\begin{document}
\title{CMS data quality monitoring: systems and experiences}
\author{L Tuura$^1$, A Meyer$^{2,3}$, I Segoni$^3$, G Della Ricca$^{4,5}$}
\address{$^1$ Northeastern University, Boston, MA, USA}
\address{$^2$ DESY, Hamburg, Germany}
\address{$^3$ CERN, Geneva, Switzerland}
\address{$^4$ INFN Sezione di Trieste, Trieste, Italy}
\address{$^5$ Universit\`a di Trieste, Trieste, Italy}
\ead{lat@cern.ch, andreas.meyer@cern.ch, ilaria.segoni@cern.ch, giuseppe.della-ricca@ts.infn.it}

\begin{abstract}
In the last two years the CMS experiment has commissioned a full end to end
data quality monitoring system in tandem with progress in the detector
commissioning.  We present the data quality monitoring and certification
systems in place, from online data taking to delivering certified data sets
for physics analyses, release validation and offline re-reconstruction
activities at Tier-1s.  We discuss the main results and lessons learnt so far
in the commissioning and early detector operation.  We outline our practical
operations arrangements and the key technical implementation aspects.
\end{abstract}

%%%%%%%%%%%%%%%%%%%%%%%%%%%%%%%%%%%%%%%%%%%%%%%%%%%%%%%%%%%%%%%%%%%%%%

\section{Overview}

Data quality monitoring (DQM) is critically important for the detector and
operation efficiency, and for the reliable certification of the recorded data
for physics analyses.  The CMS experiment at CERN's Large Hadron
Collider~\cite{cms_tp} has standardised on a single end-to-end DQM chain
(Fig.~\ref{fig:overview}).  The system comprises:

\begin{itemize}
  \item tools for the creation, filling, transport and archival of histogram
    and scalar monitor elements, with standardised algorithms for performing
    automated quality and validity tests on value distributions;
  \item monitoring systems live online for the detector, the trigger, and the
    DAQ hardware status and data throughput, for the offline reconstruction
    and for validating calibration results, software releases and simulated
    data;
  \item visualisation of the monitoring results;
  \item certification of datasets and subsets thereof for physics analyses;
  \item retrieval of DQM quantities from the conditions database;
  \item standardisation and integration of DQM components in CMS software releases;
  \item organisation and operation of the activities, including shifts and tutorials.
\end{itemize}

\begin{figure}[!b]
\begin{center}
\includegraphics[width=.90\textwidth]{Snippets/DQM_end_to_end}
\end{center}
\caption{\label{fig:overview}DQM system overview.}
\end{figure}

The high-level goal of the system is to discover and pin-point errors---%
problems occurring in detector hardware or reconstruction software---early,
with sufficient accuracy and clarity to reach good detector and operation
efficiency.  Toward this end, standardised high-level views distill the body
of quality information into summaries with significant explaining power.
Operationally CMS partitions the DQM activities in online and offline to {\em
  data processing,} {\em visualisation,} {\em certification} and {\em
  sign-off,} as illustrated in Fig.~\ref{fig:overview} and described further
in subsequent sections.  The CMS DQM supports mostly automated processes, but
use of the tools is also foreseen for the interactive and semi-automated data
processing at the CAF analysis facility~\cite{cms_caf_09}.

\begin{figure}[!b]
\begin{center}
\subfigure[Online DQM system.]{\label{fig:online}%
  \includegraphics[height=.55\textwidth]{Snippets/DQM_online}}
\hspace{1in}
\subfigure[Offline DQM system.]{\label{fig:offline}%
  \includegraphics[height=.55\textwidth]{Snippets/DQM_offline}}
\caption{\label{fig:systems}DQM workflows.}
\end{center}
\end{figure}

%%%%%%%%%%%%%%%%%%%%%%%%%%%%%%%%%%%%%%%%%%%%%%%%%%%%%%%%%%%%%%%%%%%%%%

\section{Online DQM system}
\subsection{Data processing}

As illustrated in Fig.~\ref{fig:online}, the online DQM applications in the
online are an integral part of the rest of the event data processing at the
cluster at CMS Point-5.  DQM distributions are created at two different
levels, {\em high-level trigger filter units} and {\em data quality monitoring
  applications.}

The high-level trigger filter units process events at up to 100~kHz and
produce a limited number of histograms.  The histogram monitor elements are
delivered from the filter units to the storage managers at the end of each
luminosity section.  Identical histograms across different filter units are
summed together and sent to a storage manager proxy server, which saves the
histograms to files and serves them to DQM consumer applications along with
the events.

The data quality monitoring applications receive event data and trigger
histograms from a DQM monitoring event stream from the storage manager proxy
at the rate of about 10-15~Hz, usually one application per subsystem.  Events
are filtered for the stream by applying trigger path selections specified by
the DQM group.  Each DQM application requests data specifying a subset of
those paths as a further filter.  There is no special event sorting or
handling, nor any guarantee to deliver different events to parallel DQM
applications.  The DQM stream provides raw data products only, and on explicit
request additional high level trigger information.

Each application receives events from the storage manager proxy over HTTP and
runs its choice of algorithms and analysis modules and generates its results
in the form of {\em monitoring elements,} including meta data such as the run
number and the time the last event was seen.  The applications re-run
reconstruction according to the monitoring needs.  The monitor element output
includes reference histograms and quality test results.  The latter are defined
using a generic standard quality testing module, and are configured via an XML
file.

\subsection{Visualisation}

All the result monitor element data including alarm states based on quality
test results is made available to a central DQM GUI for visualisation in real
time~\cite{dqm_gui_09}, and stored to a ROOT file~\cite{root} from time to
time during the run.  At the end of the run the final archived results are
uploaded to a large disk pool on the central GUI.  There the files are merged
to larger size and backed up to tape.  The automatic certification summary
from the online DQM step is extracted and uploaded to the run registry and on
to the condition database (see section~\ref{certification}), where it can be
analysed using another web-based monitoring tool, WBM~\cite{wbm}.  Several
months of recent DQM data is kept on disk available for archive web browsing.

\subsection{Operation}

Detector performance groups provide the application configurations to execute,
with the choice of conditions, reference histograms and the quality test
parameters to use and any code updates required.  Reviewed configurations are
deployed into a central replica {\em playback integration test system,} where
they are first tested against recent data for about 24 hours.  If no problems
appear, the production configuration is upgraded.  This practice allows CMS to
maintain high quality standard with reasonable response time, free of
artificial dead-lines.

The central DQM team invests significantly in three major areas: 1)~to
integrate and standardise the DQM processes, in particular to define and
enforce standard interfaces, naming conventions and look and feel of the
summary level information; 2)~to organise shift activities, maintain
sufficiently useful shift documentation, and train people taking shifts; and
3)~to support and consult the subsystem DQM responsibles and the physicists
using the DQM tools.

All the data processing components, including the storage manager proxy, the
DQM applications and the event display, start and stop automatically under
centralised CMS run control~\cite{runcontrol}.  The DQM GUI and WBM web
servers are long-lived server systems which are independent of the run
control.  The file management on the DQM GUI server is increasingly automated.

%%%%%%%%%%%%%%%%%%%%%%%%%%%%%%%%%%%%%%%%%%%%%%%%%%%%%%%%%%%%%%%%%%%%%%

\section{Offline DQM systems}
\subsection{Data processing}

As illustrated in Fig.~\ref{fig:overview}, numerous offline workflows in CMS
involve data quality monitoring: Tier-0 prompt reconstruction,
re-reconstruction at the Tier-1s and the validation of the alignment and
calibration results, the software releases and all the simulated data.  These
systems vary considerably in location, data content and timing, but as far as
DQM is concerned, CMS has standardised on a single two-step process for all
these activities, shown in Fig.~\ref{fig:offline}.

In the first step the histogram monitor elements are created and filled with
information from the CMS event data.  The histograms are stored as {\em run
  products} along with the processed events to the normal output event data
files.  When the CMS data processing systems merge output files together, the
histograms are automatically summed together to form the first partial result.

In a second {\em harvesting step,} run at least once at the end of the data
processing and sometimes periodically during it, the histograms are extracted
from the event data files and summed together across the entire run to yield
full event statistics on the entire dataset.  The application also obtains
detector control system (DCS, in particular high-voltage system) and data
acquisition (DAQ) status information from the offline condition database,
analyses these using detector-specific algorithms, and may create new
histograms such as high-level detector or physics object summaries.

The final histograms are used to calculate efficiencies and checked for
quality, in particular compared against reference distributions.  The
harvesting algorithms also compute the preliminary automatic {\em data
  certification decision.}  The histograms, certification results and quality
test results along with any alarms are output to a ROOT file, which is then
uploaded to a central DQM GUI web server.

The key differences between the various offline DQM processes are in content
and timing.  The Tier-0 and the Tier-1s re-determine the detector status on
real data using full event statistics and full reconstruction, and add
higher-level physics object monitoring to the detector and trigger performance
monitoring.  The Tier-0 does so at $\Delta t$ of a day or two whereas the
Tier-1 re-processing takes from days to weeks.  On CAF the time to validate
alignment and calibration quantities varies from hours to days.  The
validation cycle of simulated data reflects the sample production times, and
varies anywhere from hours for release validation to weeks on large samples.
The validation of simulated data differs from detector data in that entire
datasets are validated at once, rather than runs, and that the validation
applies to a large number of additional simulation-specific quantities.

\subsection{Visualisation}

As in the case of online, the DQM results from offline processing are uploaded
to the central DQM GUI server with a large disk pool.  There the result files
are merged to larger size and backed up to the tape; recent data is kept
cached on disk for several months.  The automatic certification results from
the harvesting, called {\em quality flags,} are extracted and uploaded to the
run registry.  From there the values are propagated to the condition database
and the dataset bookkeeping system DBS as described in
section~\ref{certification}.

CMS provides one central DQM GUI web server instance per offline activity,
including one public test instance for development.  All online and offline
servers provide a common look and feel and are linked together as one entity,
to give the entire worldwide collaboration access to inspect and analyse all
the DQM data at one central location.  While the GUI will offer in its final
form all the capabilities needed for shift and expert use for all the DQM
activities, we emphasise it is custom-built for the purpose of efficient
interactive visualisation and navigation of DQM results; it is not a general
physics analysis tool.

\subsection{Operation}

In all the offline processing the initial histogram production step is
incorporated in the standard data processing workflows as an additional
execution sequence, using components from standard software releases.  The
harvesting step implementation currently varies by the activity.  The Tier-0
processing systems has fully integrated automated harvest step and upload to
the DQM GUI.  For other data we currently submit analysis jobs with the CMS
CRAB tool~\cite{cms_crab_07} to perform the harvest step; the histogram
result file is returned in the job sandbox which the operator then uploads
to the GUI.  This is largely a manual operation at present.

%%%%%%%%%%%%%%%%%%%%%%%%%%%%%%%%%%%%%%%%%%%%%%%%%%%%%%%%%%%%%%%%%%%%%%

\begin{figure}[!tbp]
\begin{center}
\includegraphics[width=.90\textwidth]{Snippets/Run_registry}
\caption{\label{fig:runregistry}DQM run registry web interface.}
\end{center}
\end{figure}

\section{Certification and sign-off workflow}\label{certification}

CMS uses a {\em run registry} database with a front-end web application as the
central workflow tracking and bookkeeping tool to manage the creation of the
final physics dataset certification result.  The run registry is both a user
interface managing the workflow (Fig.~\ref{fig:runregistry}), and a persistent
store of the information; technically speaking it is part of the online
condition database and the web service is hosted as a part of the WBM system.

The certification process begins with the physicists on online and offline
shift filling in the run registry with basic run information, and adding any
pertinent observations on the run during the shift.  This information is then
augmented with the automatic data certification results from the online,
Tier-0 and Tier-1 data processing as described in the previous sections.  This
results in basic major detector segment level certification which accounts for
DCS, DAQ and DQM online and offline metrics, and in future may also include
power and cooling status.  For each detector segment and input one single
boolean flag or a floating point value describes the final quality result.
For the latter we apply appropriate thresholds which yield binary ``good'' or
``bad'' results.  We label the result ``unknown'' if no quality flag was
calculated.

Once the automatic certification results are known and uploaded to the run
registry, the person on shift evaluates the detector and physics object
quality following the shift instructions on histograms specifically tailored
to catch relevant problems.  This person adds any final observations to the
run registry and overrides the automatic result with a manual certification
decision where necessary.

The final combined quality result is then communicated to the detector and
physics object groups for confirmation.  Regular sign-off meetings collect the
final verdict and deliver the agreed result to the experiment.  At this stage
the quality flags are copied to the offline condition database and to the
dataset bookkeeping system (DBS)~\cite{cms_dbs_07}.  The flags are stored in
conditions as data keyed to an interval of validity, and are meant for use in
more detailed filtering in any subsequent data processing, longer-term data
quality evaluation, and correlation with other variables such as temperature
data.  In the DBS the quality flags are used to define convenience {\em
  analysis datasets.}  The flags are also accessible in the CMS data discovery
interface~\cite{cms_dbs_discovery_07}, a web interface to browse and select
data in the DBS (Fig.~\ref{fig:dbsqflags}).

Some trivial trend plots of the key monitor element metrics have recently
been generated automatically.  We plan to extend this to more comprehensive
interactive trend plotting of any selected histogram metric, and are working
on common design for convenient access and presentation of trends over time.

\begin{figure}[!tbp]
\begin{center}
\includegraphics[width=.90\textwidth]{Snippets/DBS_discovery_qflags}
\caption{\label{fig:dbsqflags}DBS discovery page displaying quality flags.}
\end{center}
\end{figure}

%%%%%%%%%%%%%%%%%%%%%%%%%%%%%%%%%%%%%%%%%%%%%%%%%%%%%%%%%%%%%%%%%%%%%%

\section{Organisation and operation}

Online shifts take place 24/7 during detector operation at the CMS Point-5 in
Cessy.  Offline DQM shifts are carried out in day time at the CMS centre~%
\cite{cms_centres_09} on the main CERN site.  The shift activities are
supported by regular remote shifts, two shifts per day at Fermilab and one
shift per day at DESY, at the local CMS centre~%
\cite{collaboration_at_distance_09}.  Standard shift instructions, as
illustrated in Fig.~\ref{fig:shiftdoc}, have been fully exercised.

\begin{figure}[!tbp]
\begin{center}
\includegraphics[width=.75\textwidth]{Snippets/DQM_shift_instructions}
\caption{\label{fig:shiftdoc}Example shift instructions.}
\end{center}
\end{figure}

%%%%%%%%%%%%%%%%%%%%%%%%%%%%%%%%%%%%%%%%%%%%%%%%%%%%%%%%%%%%%%%%%%%%%%

\section{Experience}

CMS has commissioned a full end to end data quality monitoring system in
tandem with the detector over the last two years.  The online DQM system has
now been in production for about a year and the full offline chain has just
been commissioned: we have recently completed the first full cycle of
certification and sign-offs.  DQM for the less structured alignment and
calibration activity at the CAF exists but a fair amount of work remains.

In our experience it takes approximately one year to commission a major
component such as online or offline DQM to production quality.  Shift
organisation, instructions, tutorials and supervision are major undertakings.
Significant amounts of effort are needed in various groups to develop the DQM
algorithms, and centrally to standardise and integrate the workflows,
procedures, code, systems and servers.  While we find only modest amounts of
code are needed for the core DQM systems, on the balance there is a perpetual
effort to optimise histograms to maximise sensitivity to problems, to
standardise the look and feel and to improve efficiency through better
documentation, and a battle to promote sharing and use of common code against
natural divergence in a collaboration as large as CMS.

CMS has so far focused on commissioning a common first order DQM system
throughout the entire experiment, with the aim of having an effective basic
system ready for the first beam.  We believe we have successfully achieved
this goal and will address second order features in due course.

CMS is very pleased with the DQM visualisation served using web technology and
operating shifts from the CMS centres.  Remote access to all the DQM
information, especially offsite real-time live access to the online as been
greatly beneficial and appreciated.  Together the CMS centres and remote
access have been essential and practical enabling factors to the experiment.

%%%%%%%%%%%%%%%%%%%%%%%%%%%%%%%%%%%%%%%%%%%%%%%%%%%%%%%%%%%%%%%%%%%%%%

\ack

The authors thank the numerous members of CMS collaboration who have
contributed the development and operation of the DQM system.  Effective data
quality monitoring is a truly collaborative effort involving a lot of people
from several other projects: the trigger, detector subsystems, offline and
physics software, production tools, operators, and so on.
 
%%%%%%%%%%%%%%%%%%%%%%%%%%%%%%%%%%%%%%%%%%%%%%%%%%%%%%%%%%%%%%%%%%%%%%

\section*{References}
\begin{thebibliography}{9}

\bibitem{cms_tp}
  CMS Collaboration,
  1994, {\it CERN/LHCC 94-38},
  ``Technical proposal''
  (Geneva, Switzerland)

\bibitem{dqm_gui_09}
  Tuura L, Eulisse G, Meyer A,
  2009, {\it Proc. CHEP09, Computing in High Energy Physics},
  ``CMS data quality monitoring web service''
  (Prague, Czech Republic)

\bibitem{cms_caf_09}
  Kreuzer P, Gowdy S, Sanches J, et al,
  2009, {\it Proc. CHEP09, Computing in High Energy Physics},
  ``Building and Commissioning of the CMS CERN Analysis Facility (CAF)''
  (Prague, Czech Republic)

\bibitem{root}
  ROOT---A data analysis framework, 2009, \url{http://root.cern.ch}

\bibitem{runcontrol}
  Bauer G, et al,
  2007, {\it Proc. CHEP07, Computing in High Energy Physics},
  ``The Run Control System of the CMS Experiment''
  (Victoria B.C., Canada)

\bibitem{wbm}
  Badgett W,
  2007, {\it Proc. CHEP07, Computing in High Energy Physics},
  ``CMS Online Web-Based Monitoring and Remote Operations''
  (Victoria B.C., Canada)

\bibitem{cms_crab_07}
  Spiga D,
  2007, {\it Proc. CHEP07, Computing in High Energy Physics},
  ``CRAB (CMS Remote Anaysis Builder)'',
  (Victoria B.C., Canada)

\bibitem{cms_dbs_07}
  Afaq A, Dolgbert A, Guo Y, Jones C, Kosyakov S,
  Kuznetsov V, Lueking L, Riley D, Sekhri V,
  2007, {\it Proc. CHEP07, Computing in High Energy Physics},
  ``The CMS Dataset Bookkeeping Service''
  (Victoria B.C., Canada)

\bibitem{cms_dbs_discovery_07}
  Dolgert A, Gibbons L, Kuznetsov V, Jones C, Riley D,
  2007, {\it Proc. CHEP07, Computing in High Energy Physics},
  ``A multi-dimensional view on information retrieval of CMS data''
  (Victoria B.C., Canada)

\bibitem{cms_centres_09}
  Taylor L, Gottschalk E,
  2009, {\it Proc. CHEP09, Computing in High Energy Physics},
  ``CMS Centres Worldwide: a New Collaborative Infrastructure''
  (Prague, Czech Republic)

\bibitem{collaboration_at_distance_09}
  Gottschalk E,
  2009, {\it Proc. CHEP09, Computing in High Energy Physics},
  ``Collaborating at a Distance: Operations Centres, Tools, and Trends''
  (Prague, Czech Republic)

\end{thebibliography}
\end{document}
