\documentclass[a4paper]{jpconf}
\usepackage[dvipdfm]{graphicx}
\usepackage[dvipdfm]{hyperref}
\usepackage{subfigure}
\usepackage{mediabb}
\begin{document}
\title{CMS data quality monitoring web service}
\author{L Tuura$^1$, G Eulisse$^1$, A Meyer$^{2,3}$}
\address{$^1$ Northeastern University, Boston, MA, USA}
\address{$^2$ DESY, Hamburg, Germany}
\address{$^3$ CERN, Geneva, Switzerland}
\ead{lat@cern.ch, giulio.eulisse@cern.ch, andreas.meyer@cern.ch}

\begin{abstract}
A central component of the data quality monitoring system of the CMS
experiment at the Large Hadron Collider is a web site for browsing
data quality histograms.  The production servers in data taking
provide access to several hundred thousand histograms per run, both
live in online as well as for up to several terabytes of archived
histograms for the online data taking, Tier-0 prompt reconstruction,
prompt calibration and analysis activities, for re-reconstruction at
Tier-1s and for release validation.  At the present usage level the
servers currently handle in total around a million authenticated HTTP
requests per day.  We describe the main features and components of the
system, our implementation for web-based interactive rendering, and
the server design.  We give an overview of the deployment and
maintenance procedures.  We discuss the main technical challenges and
our solutions to them, with emphasis on functionality, long-term
robustness and performance.
\end{abstract}

%%%%%%%%%%%%%%%%%%%%%%%%%%%%%%%%%%%%%%%%%%%%%%%%%%%%%%%%%%%%%%%%%%%%%%

\section{Overview}

CMS~\cite{cms_tp} developed the DQM GUI, a web-based user interface
for visualising data quality monitoring data for two reasons.  For
one, it became evident we would much prefer a web application over a
local one~\cite{dqm_sistrip_07,dqm_ajax_06,cms_webtools_07}~%
(Fig.~\ref{fig:arch}).  Secondly, we wanted a single customisable
application capable of delivering visualisation for all the DQM needs
in all of CMS, for all subsystems, for live data taking as much as
archives and offline workflows~\cite{dqm_overview_09}.

Content is exposed as {\em workspaces} (Fig.~\ref{fig:views}) from
high-level summaries to shift views to expert areas, including even a
basic histogram style editor.  Event display snapshots are also
accessible.  The server configuration specifies the available
workspaces.

Within a workspace histograms can be organised into {\em layouts} to
bundle related information together.  A layout defines not only the
composition, but can also provide documentation
(Fig.~\ref{fig:views-reduced-histos},~\ref{fig:views-ecal-shift}),
change visualisation settings, and for example turn the reference
histogram display on (Fig.~\ref{fig:views-reference}).  Shift views
are usually defined as collections of layouts.

\begin{figure}[!tbp]
\begin{minipage}[b]{.40\textwidth}
\vspace{0em}
\begin{center}
\raisebox{3em}{
  \includegraphics[width=.90\textwidth]{Snippets/DQM_GUI_architecture_v2}}
\end{center}
\caption{\label{fig:arch}GUI architecture overview.}
\end{minipage}
\hfill
\begin{minipage}[b]{.55\textwidth}
\vspace{0em}
\begin{center}
\tiny
\begin{verbatim}
([{kind: 'AutoUpdate', interval: 300, stamp: 1237219783, serverTime: 96.78},
 { kind: 'DQMHeaderRow', run: "77'025", lumi: "47", event: "6'028'980",
   service: 'Online', workspace: 'Summary', page: 1, pages: 1, services: [...],
   workspaces: [{title: 'Summary', label: 'summary',
                 category: 'Summaries', rank: 0}, ...],
   runs: ["Live", "77057", ...], runid: 77025},
 { kind: 'DQMQuality', items: [
   { label: "CSC", version:1236278233000000000,
     name: "CSC/EventInfo/reportSummaryMap",
     location: "archive/77025/Global/Online/ALL/Global run",
     reportSummary: "0.998", eventTimeStamp: "1236244352" },  ...]}])
\end{verbatim}
\end{center}
\caption{\label{fig:json}A JSON state response from which the Summary
  page of Fig.~\ref{fig:views-summary-page} was rendered.  The
  response contains the minimal raw data needed for this particular
  page view, grouped by browser side JavaScript GUI plug-in which
  knows how to translate the state in HTML+CSS, including the user
  interaction controls.}
\end{minipage}
\end{figure}

\begin{figure}[!tbp]
\begin{center}
\subfigure[General summary.]{\label{fig:views-summary-page}%
  \includegraphics[width=.31\textwidth]{Snippets/DQM_GUI_summary_view}}\hfill
\subfigure[Reduced histograms in normal browsing mode, rendered with anti-aliasing.]{\label{fig:views-reduced-histos}%
  \includegraphics[width=.31\textwidth]{Snippets/DQM_GUI_everything}}\hfill
\subfigure[An event display workspace.]{\label{fig:views-event-display}%
  \includegraphics[width=.28\textwidth]{Snippets/DQM_GUI_with_event_display}}\\
\subfigure[Reference histogram drawn.]{\label{fig:views-reference}%
  \includegraphics[width=.31\textwidth]{Snippets/DQM_GUI_reference}}\hfill
\subfigure[The standard ECAL barrel shift workspace.]{\label{fig:views-ecal-shift}%
  \includegraphics[width=.31\textwidth]{Snippets/DQM_GUI_shift_view}}\hfill
\subfigure[Single histogram explorer and editor.]{\label{fig:views-histo-edit}%
  \includegraphics[width=.31\textwidth]{Snippets/DQM_GUI_plot_edit}}
\end{center}
\caption{\label{fig:views}DQM GUI views.}
\end{figure}

%%%%%%%%%%%%%%%%%%%%%%%%%%%%%%%%%%%%%%%%%%%%%%%%%%%%%%%%%%%%%%%%%%%%%%

\section{Implementation}

Our server is built on CherryPy, a Python language web server
framework~\cite{cherrypy}.  The server configuration and the HTTP API
are implemented in Python.  The core functionality is in a C++
accelerator extension.  The client is a GUI-in-a-browser, written
entirely in JavaScript.  It fetches content from the server with
asynchronous calls, a technique known as
AJAX~\cite{dqm_ajax_06,cms_webtools_07}.  The server responds in
JSON~\cite{dqm_ajax_06,json}.  The browser code forms the GUI by
mapping the JSON structure to suitable HTML+CSS content
(Fig.~\ref{fig:json},~\ref{fig:views}).

The user session state and application logic are held entirely on the
web server; the browser application is ``dumb.'' User's actions such
as clicking on buttons are mapped directly to HTTP API calls such as
{\tt setRun?v=123}.  The server responds to API calls by sending back
a new full state in JSON, but only the minimum required to display the
page at hand.  The browser compares the new and the old states and
updates the page incrementally.  This arrangement trivially allows one
to reload the web page, to copy and paste URLs, or to resume the
session later.

The server responds to most requests directly from memory, yielding
excellent response time.  All tasks which can be delayed are handled
in background threads, such as receiving data from the network or
flushing session state to disk.  The server data structures support
parallel traversal, permitting several HTTP requests to be processed
concurrently.

\begin{figure}[!tbp]
\begin{center}
\includegraphics[width=.45\textwidth]{Snippets/DQM_GUI_custom_original}\hfill
\raisebox{6em}{\LARGE\,$\Rightarrow$}\hfill
\includegraphics[width=.45\textwidth]{Snippets/DQM_GUI_custom_render}
\end{center}
\caption{\label{fig:plugin}A render plug-in can be added to modify the appearance of a histogram.}
\end{figure}

\begin{figure}[!tbp]
\begin{minipage}{.45\textwidth}
\begin{center}
\includegraphics[width=.99\textwidth]{Snippets/CMS_centre_console_small_extract}
\caption{\label{fig:console}One of the central DQM and event display consoles at the CMS centre in Meyrin, 10 Sept 2008.}
\end{center}
\end{minipage}
\hfill
\begin{minipage}{.45\textwidth}
\begin{center}
\includegraphics[width=.70\textwidth]{Snippets/DQM_GUI_servers}
\caption{\label{fig:shmem}The distributed shared memory system.}
\end{center}
\end{minipage}
\end{figure}

The histograms are rendered in separate fortified processes to isolate
the web server from ROOT's instability.  The server communicates with
the renderer via low-latency distributed shared memory.  Live DQM data
also resides in distributed shared memory (Fig.~\ref{fig:shmem}).
Each producer hosts its own histograms and notifies the server about
updates.  The renderer retrieves histograms asynchronously from the
producers on demand; although single-threaded for ROOT, it can have
dozens of operations in progress concurrently.  Recently accessed
views are re-rendered automatically on histogram update to reduce the
image delivery latency.

Our DQM data is archived in ROOT files~\cite{root}.  As reading ROOT
files in the web server itself would be too slow, use too much memory,
prone to crash the server, and would seriously limit concurrency, we
index the ROOT data files on upload.  The GUI server computes its
response using only the index.  The ROOT files are accessed only to
get the histograms for rendering in a separate process
(Fig.~\ref{fig:shmem}).  The index is currently a simple SQLite
database~\cite{sqlite}.

The server supports setting basic render options, such as linear
vs. log axis, axis bounds and ROOT draw options
(Fig.~\ref{fig:views-histo-edit}).  These settings can be set
interactively or as defaults in the subsystem layout definitions.  The
subsystems can further customise the default look and feel of the
histograms by registering C++ render plug-ins, which are loaded on
server start-up (Fig.~\ref{fig:plugin}).  We improve image quality
significantly by generating images in ROOT in much larger size than
requested, then reducing the image to the final smaller size using a
high-quality anti-alias filter.

%%%%%%%%%%%%%%%%%%%%%%%%%%%%%%%%%%%%%%%%%%%%%%%%%%%%%%%%%%%%%%%%%%%%%%

\begin{figure}[!b]
\begin{center}
\subfigure[The daily average DQM GUI response time versus date and number of requests per day.]{\label{fig:responsetime}%
  \includegraphics[width=.31\textwidth]{Snippets/DQM_GUI_accesses_response}}\hfill
\subfigure[Number of DQM GUI HTTP requests per day from December 2007 to March 2009.]{\label{fig:requests}%
  \includegraphics[width=.65\textwidth]{Snippets/DQM_GUI_accesses_traffic}}\hfill
\end{center}
\caption{\label{fig:performance}DQM GUI HTTP server performance.}
\end{figure}

\section{Operation and experience}

CMS centrally operates four DQM GUI instances for online and offline
each, an instance per purpose for the existence of data: Tier-0, CAF,
release validation, and so on.  In addition at least four instances
are operated by detector subsystems in online for exercises private to
the detector.  Most DQM developers also run a private GUI instance
while testing.  A picture of a live station is shown in
Fig.~\ref{fig:console}.

Early on it became abundantly evident ROOT was neither robust nor
suitable for long-running servers.  Some three quarters of all the
effort on the entire DQM GUI has gone into debugging ROOT
instabilities and producing countermeasures.  We are very pleased with
the robustness of the rest of the DQM GUI system.

CMS typically creates circa 50'000 histograms per run.  The average
GUI HTTP response time is around 270 ms (Fig.~\ref{fig:responsetime}),
which we find satisfactory.  The production server has scaled to about
750'000 HTTP requests per day with little apparent impact on the
server response time (Fig.~\ref{fig:requests}).  Interestingly the
vast majority of the accesses are to the online production server from
outside the online environment.  This indicates the web-based
monitoring and visualisation solution applies well to the practical
needs of the experiment.  We have exercised the GUI with up to 300'000
histograms per run.  The GUI remains usable although there is a
perceivable interaction delay.  We plan to optimise the server further
such that it has ample capacity to gracefully handle growing histogram
archives and special studies with large numbers of monitored entities.

%%%%%%%%%%%%%%%%%%%%%%%%%%%%%%%%%%%%%%%%%%%%%%%%%%%%%%%%%%%%%%%%%%%%%%

\ack

The authors thank the numerous members of CMS collaboration who
provided abundant feedback and ideas on making the GUI server more
effective and useful for the experiment.

%%%%%%%%%%%%%%%%%%%%%%%%%%%%%%%%%%%%%%%%%%%%%%%%%%%%%%%%%%%%%%%%%%%%%%

\section*{References}
\begin{thebibliography}{9}

\bibitem{cms_tp}
  CMS Collaboration,
  1994, {\it CERN/LHCC 94-38},
  ``Technical proposal''
  (Geneva, Switzerland)

\bibitem{dqm_sistrip_07}
  Giordano D et al,
  2007, {\it Proc. CHEP07, Computing in High Energy Physics},
  ``Data Quality Monitoring for the CMS Silicon Strip Tracker''
  (Victoria, Canada)

\bibitem{dqm_ajax_06}
  Eulisse G, Alverson G, Muzaffar S, Osborne I, Taylor L and Tuura L,
  2006, {\it Proc. CHEP06, Computing in High Energy Physics},
  ``Interactive Web-based Analysis Clients using AJAX: with examples
  for CMS, ROOT and GEANT4''
  (Mumbai, India)

\bibitem{cms_webtools_07}
  Metson S, Belforte S, Bockelman B, Dziedziniewicz K, Egeland R,
  Elmer P, Eulisse G, Evans D, Fanfani A, Feichtinger D, Kavka C,
  Kuznetsov V, van Lingen F, Newbold D, Tuura L and Wakefield S,
  2007, {\it Proc. CHEP07, Computing in High Energy Physics},
  ``CMS Offline Web Tools''
  (Victoria, Canada)

\bibitem{dqm_overview_09}
  Tuura L, Meyer A, Segoni I and Della Ricca G,
  2009, {\it Proc. CHEP09, Computing in High Energy Physics},
  ``CMS data quality monitoring: systems and experiences''
  (Prague, Czech Republic)

\bibitem{cherrypy}
  CherryPy---A pythonic, object-oriented HTTP framework, 2009,
  \url{http://cherrypy.org}

\bibitem{json}
  Introducing JSON, 2009, \url{http://json.org}

\bibitem{root}
  ROOT---A data analysis framework, 2009, \url{http://root.cern.ch}

\bibitem{sqlite}
  SQLite---A library implementing a self-contained SQL database engine,
  2009, \url{http://sqlite.org}

\end{thebibliography}
\end{document}
