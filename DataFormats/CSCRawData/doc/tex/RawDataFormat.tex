\def\Hline{\\ \hline}
\def\HHline{\\ \hline\hline}
\def\half{\mathrel{\frac{1}{2}}}
\def\onethird{\mathrel{\frac{1}{3}}}
\def\twothirds{\mathrel{\frac{2}{3}}}
\def\TeV{~TeV}

% A tabular with the specified number of bits (with no leading field)
\newenvironment{DDUtabular}[2][3.5mm]
{\vspace{2mm}\hspace{-0.1\linewidth}\begin{minipage}[t]{1.2\linewidth}
\begin{center}
\small\tabcolsep0.2mm
\begin{tabular}{|*{#2}{m{#1}<{\centering}|}}\hline}
{\end{tabular}\end{center}\end{minipage}}

% The first field of the row spanning several bits
\newcommand{\ffield}[2]{\multicolumn{#1}{|c|}{ #2}}

% A double-row field spanning several bits
\newcommand{\dfield}[2]{\footnotesize\multicolumn{#1}{c{3.5mm}|}{ #2}}

% A tiny text, double-row field spanning several bits
\newcommand{\tfield}[2]{\multicolumn{#1}{c|}{\parbox{6.8mm}{\centering\tiny #2}}}

% A tabular with the specified number of bits (plus one leading word# field)
\newenvironment{TMBtabular}[2][8.5mm]
{\vspace{2mm}\hspace{-0.1\linewidth}\begin{minipage}[t]{1.2\linewidth}
\begin{center}
\small\tabcolsep0.2mm
\begin{tabular}{|c||*{#2}{m{#1}<{\centering}|}}\hline}
{\end{tabular}\end{center}\end{minipage}}
 


\section{CSC Data Format}\label{sec:CSC}

The Endcap Muon (EMU) system consists of 468 Cathode Strip Chambers (CSCs),
each providing multiple data sources in the data stream:
% every CSC has several subcomponents within its datastream.
%The Endcap Muon (EMU) system is composed of 468 Cathode Strip Chambers (CSCs);
% every CSC has several subcomponents within its datastream.
% Every CSC has its own Data Acquisition Motherboard (DMB) to read out
% and buffer the Cathode and Anode raw data and trigger primitives
% sent from these sources:
\begin{itemize}
\item Data Acquisition Motherboard (1 DMB per CSC)
\item Cathode Front End Board (4-5 CFEBs per CSC)
\item Anode Front End Board (12-42 AFEBs per CSC)
\item Trigger Motherboard (1 TMB per CSC)
\item Anode LCT Board (1 ALCT per CSC)
\end{itemize}
The DMB processes the data from each source in sequence on an event-by-event
basis to form a data packet that is transmitted
% This data is sent 
via fiber to a DDU in an EMU FED crate.
The 36 DDUs in the EMU system are read out through 4 DCCs with a total of
8 S-Link64 mezzanine boards.  There is a 37th DDU for the CSC Trackfinder
that has its own dedicated S-Link64 readout to the FRL.

Each component mentioned above has a unique data format that is
detailed in the following sections.

\subsection{EMU DCC Format}

All CSC data is read out through 4 Data Concentrator Cards
via 8 S-Link64 mezzanine boards.  
Each S-Link carries data from 4 to 5 EMU DDUs wrapped by DCC Headers
and Trailers as shown below:
\begin{itemize}
\item   DCC Header block (2 64-bit words)
\item   DDU-0 Data  (variable number of 64-bit words)
\item   DDU-1 Data
\item   DDU-2 Data
\item   DDU-3 Data
\item   DDU-4 Data  (present in 4 out of 8 EMU S-Links)
\item   DCC Trailer block (2 64-bit words)
\end{itemize}
The DCC Header and Trailer blocks conform to the S-Link protocol in
the CMS Common Data Format.  EMU-specific event information is carried
in the Second Header and First Trailer, as shown in
Table~\ref{tab:EmuDCCformat}, which carry the following fields:
\begin{description}
\item {\bf Orbit\_Count}: Number of LHC orbits since last L1 Reset.
\item {\bf DCC\_FIFO\_Status}: 5-bit FIFOs\_Used register, 
5-bit FIFOs\_Almost\_Empty register, 5-bit FIFOs\_Empty register, 1 dummy bit.
\item {\bf Proc\_Time}: Time used to process the current event, encoded in
8-bits: '$a b c d e f g h$' converts to $(bcdefgh)*(16**a)*0.41$~us.
\item {\bf DDU\_data\_status}: DCC Input FIFO status register, 8-bits per DDU.
\item {\bf Timeout\_Flags}: Timeout register, 1-bit per DDU plus 3 dummy bits.
\end{description}

% 64-bit table, in a landscape-rotated page
\begin{landscape}
\thispagestyle{empty}
\begin{table}[htbp]
  \caption{EMU DCC implementation of Common FED data format.}\label{tab:EmuDCCformat}
    \begin{bittabular}{64}
      & \bitNumEightByte
      \bitline{K & 
        \field{4}{BOE\_1} & 
        \field{4}{Evt\_ty} &
        \field{24}{LV1\_id} & 
        \field{12}{BX\_id} & 
        \field{12}{Source\_id} &
        \field{4}{FOV} &
        H & x & \$ & \$}
      \bitline{ K & 
	\field{8}{D9h} & 
	\field{32}{Orbit\_Count} & 
	\field{16}{DCC\_FIFO\_Status} & 
	\field{8}{17h} }
      \bitline{ D & 
	\field{64}{DDU\_Data\_Payload} }
      \vdotline{64}
      \bitline{ K & 
	\field{8}{EFh} & 
	\field{8}{Proc\_Time} & 
	\field{40}{DDU\_data\_status (8-bits each)} & 
	\field{8}{Timeout\_Flags} }
      \bitline{K & 
        \field{4}{EOE\_1} & 
        \field{4}{xxxx} &
        \field{24}{Evt\_lgth} & 
        \field{16}{CRC} & 
        \field{4}{xxxx} &
        \field{4}{EVT\_stat} &
        \field{4}{TTS} & T & x & \$ & \$}
      & \bitNumEightByte
    \end{bittabular}
\end{table}
\vspace{.5in}
\begin{table}[hbp]
  \caption{EMU DDU data format.}\label{tab:EmuDDUformat}
    \begin{DDUtabular}{64}
      \bitNumEightByte
      \bitline{
        \ffield{4}{BOE\_1} & 
        \field{4}{Evt\_ty} &
        \field{24}{LV1\_id} & 
        \field{12}{BX\_id} & 
        \field{12}{Source\_id} &
        \field{4}{FOV} &
        H & x & \$ & \$}
      \bitline{
	\ffield{16}{8000h} & 
	\field{16}{0001h} & 
	\field{16}{8000h} & 
	\field{1}{0} & 
	\field{15}{CSCs\_Full (1-bit per CSC)} }
      \bitline{
	\ffield{1}{0} & 
	\field{15}{CSCs\_alive (1-bit per CSC)} & 
	\field{16}{DDU\_output\_status (see Table~\ref{tab:EmuDDUostat})} & 
	\field{1}{0} & 
	\field{15}{CSC\_data\_available (1-bit per CSC)} & 
	\field{1}{0} & 
	\field{7}{BoE\_status$^*$} & 
% \footnote{See Table~\ref{tab:EmuDDUboestat}.}
	\field{4}{BoE\_TTS} & 
	\field{4}{CSC\_count} }
% 0   CSCs\_Full -15  CSCs\_alive -15   0   CSC\_data\_available -15
% Output\_status -16   0   BoE\_status   BoE\_TTS   CSC\_count
% EOE\_status -32   Q   CSC\_error -15   0   CSC\_warn -15
      \bitline{
	\ffield{64}{CSC\_Data\_Payload} }
      \vdotline{64}
      \bitline{
	\ffield{16}{8000h} & 
	\field{16}{FFFFh} & 
	\field{16}{8000h} & 
	\field{16}{8000h} }
      \bitline{
	\ffield{32}{DDU\_Event\_status (see Table~\ref{tab:EmuDDUstat})} & 
	\field{1}{Q} & 
	\field{15}{CSC\_error (1-bit per CSC)} & 
	\field{1}{0} & 
	\field{15}{CSC\_warn (1-bit per CSC)} }
      \bitline{
        \ffield{4}{EOE\_1} & 
        \field{4}{xxxx} &
        \field{24}{Evt\_lgth} & 
        \field{16}{CRC} & 
        \field{4}{xxxx} &
        \field{4}{EVT\_stat} &
        \field{4}{TTS} & T & x & \$ & \$}
      \bitNumEightByte
    \end{DDUtabular}
\end{table}

$^*$ \, See Table~\ref{tab:EmuDDUboestat}. 

Q: \, CSC Full flag.
\end{landscape}

\subsection{EMU DDU Format}
The EMU DDU takes in data from up to 15 EMU DMBs on 1.6 Gb/s fiber
optic cables (1 fiber per CSC) via high-density (LC) front panel inputs.
The data from each fiber gets deserialized by the Rocket IO within a Xilinx
FPGA to recover the original 16-bit DMB data structure, which is then
buffered in a FIFO array.  Each DMB sends a multiple of 4 16-bit words
for every event, thus the DDU performs all processing on a 64-bit word basis.

Upon receipt of an L1A trigger, the DDU
%cycles through the stored data for 15 CSCs in order (0 to 14)
%and sends out the entire event record
reads the complete event record from each of the CSCs in
order (0 to 14) and sends the data
to the EMU DCC via pairs of multi-gigabit lines embedded in
the custom FED backplane.
There is a parallel SPY path for data monitoring via gigabit Ethernet
that can be prescaled as needed;
this path has the same DDU data format as the primary
readout path to the DCC.
The DDU format conforms to the CMS Common Data format, except that
the 64-bit structure does not allow the K-bit to be
set at the DDU level.  However, there is a unique bit sequence embedded
in the second DDU Header and the first DDU Trailer to serve as markers
in the data stream, as shown in Table~\ref{tab:EmuDDUformat}.  Other fields
include:
\begin{description}
\item {\bf CSCs\_Full}: Shows any CSCs in a full state.
\item {\bf CSCs\_Alive}: Shows the DDU fiber inputs that are connected to active CSCs.
\item {\bf CSC\_data\_available}: Shows which of the active CSCs have
track-stub data for this event.
\item {\bf CSC\_count}: The number of active CSCs with data for this event.
\item {\bf BoE\_TTS}: The DDU TTS status at the beginning of event.
\item {\bf Q}: Set high when a CSC has reached a full state.
\item {\bf CSC\_error}: Shows any active CSCs in an error state.
\item {\bf CSC\_warn}: Shows any active CSCs in a warning state.
\end{description}

During event processing the DDU %performs an extensive series of checks for
monitors the incoming data to check
the synchronization, word counts and CRCs, as well as
warning and error conditions that may be set for the CSC.
% coherence
After the sync check is completed, any CSCs with no muon track stubs are
excluded by zero-suppression logic, and those with track stubs are flagged
in the ``CSC\_data\_available'' field of the DDU Header.
A summary of results from all DDU checks is included in the DDU headers
and trailers that wrap each event fragment (see Tables~\ref{tab:EmuDDUformat}
through \ref{tab:EmuDDUstat}).

\begin{table}[tbp]
  \caption{EMU DDU Output Status Bit Definitions.}\label{tab:EmuDDUostat}
  \begin{center}
  \begin{tabular}{|c|l|c|}
    \hline
    Bit   &   Output Status Description & Condition\HHline
    15 & DDU Output-Limited Buffer Overflow occurred: & may be OK\\
       & \hspace{.1in} DCC/S-Link or SPY path was rate limited resulting in a buffer overflow & \\
    14 & DAQ Wait was asserted by S-Link or DCC & OK\\
    13 & Link Full (LFF) was asserted by DDU S-Link & OK\\
    12 & DDU S-Link Never Ready & OK\\
    11 & SPY FIFO Overflow occurred & OK\\
    10 & SPY Event was skipped to prevent overflow & OK\\
    9 & SPY FIFO Always Empty & OK\\
    8 & SPY Fiber Connection Error occurred & OK\Hline
    7 & DDU Buffer Overflow caused by DAQ Wait: & RESET\\
      & \hspace{.1in} DCC/S-Link was rate-limited resulting in a buffer overflow & \\
    6 & DAQ Wait is set by DCC/S-Link & OK\\
    5 & Link Full (LFF) is set by DDU S-Link & OK\\
    4 & Not Ready is set by DDU S-Link & OK\\
    3 & SPY FIFO is Full & OK\\
    2 & SPY Path was Not Enabled for this event: & OK\\
      & \hspace{.1in} no SPY data was sent due to prescale or throughput limitation & \\
    1 & SPY FIFO is Not Empty & OK\\
    0 & DCC Link is Not Ready & may be OK\Hline
  \end{tabular}
  \end{center}
\end{table}
\begin{table}[tbp]
  \caption{EMU DDU Beginning of Event Status Bit Definitions.}\label{tab:EmuDDUboestat}
  \begin{center}
  \begin{tabular}{|c|l|c|}
    \hline
    Bit   &   BoE Status Description & Condition\HHline
    6 & DDU Clock-DLL Error:\hspace{.1in}the DDU lost it's clock for an unknown period of time; & may be OK\\
      & \hspace{.1in}some data may be lost & \\
    5 & DDU FIFO Full Error:\hspace{.1in}an unknowable amount of data has been lost & RESET\\
    4 & DDU detected Fiber Error:\hspace{.1in}bad fiber connection status & RESET\\
    3 & DDU detected Critical Error:\hspace{.1in}OR of all "RESET" cases, held until RESET & RESET\\
    2 & DDU detected Single Error:\hspace{.1in}OR of all "Bad Event" cases at BoE & RESET\\
    1 & DDU detected Single Error:\hspace{.1in}DDU L1A event number match failed for 1 or more CSCs & RESET\\
    0 & DDU Start Timeout Error:\hspace{.1in}data from a CSC never arrived & RESET\Hline
  \end{tabular}
  \end{center}
\end{table}
\begin{table}[hbp]%[t]
  \caption{EMU DDU Event Status Bit Definitions.}\label{tab:EmuDDUstat}
  \begin{center}
  \begin{tabular}{|c|l|c|}
    \hline
    Bit   &   Event Status Description & Condition\HHline
    31 & CSC LCT/DAV Mismatch occurred:\hspace{.1in}CSC trigger and data do not match & bad event \\
    30 & DDU CFEB L1 Number Mismatch occurred & bad event \\
    29 & DDU detected no good DMB CRCs:\hspace{.1in}empty event or bit-error & may be bad event \\
    28 & DDU CFEB Count Error occurred:\hspace{.1in}missing CFEB data & bad event \\
    27 & DDU Bad First Data Word from CSC:\hspace{.1in}RX or bit-error & bad event \\
    26 & DDU L1A-FIFO Full Error & RESET \\
    25 & DDU Data Stuck in FIFO Error:\hspace{.1in}data not associated with a trigger, sync error & RESET \\
    24 & DDU NoLiveFibers Error:\hspace{.1in}the DDU detects no CSC fibers & may be OK \Hline
    23 & DDU Control Word Inconsistency Warning:\hspace{.1in}RX or bit-error & may be bad event \\
    22 & DDU Input FPGA Error & bad event \\
    21 & DCC Stop Bit set & OK \\
    20 & DCC Link Not Ready & may be OK \\
    19 & DDU detected TMB Error & bad event \\
    18 & DDU detected ALCT Error & bad event \\
    17 & DDU detected TMB or ALCT Word Count Error & bad event \\
    16 & DDU detected TMB or ALCT L1A Number Error & bad event \Hline
    15 & DDU detected Critical Error:\hspace{.1in}OR of all "RESET" cases, held until RESET & RESET \\
    14 & DDU detected Single Error:\hspace{.1in}OR of all "Bad Event" cases & bad event \\
    13 & DDU Single Warning:\hspace{.1in}OR of bits 23,10 & may be bad event \\
    12 & DDU FIFO Near Full Warning:\hspace{.1in}OR of ``near Full'' from all sources & OK \\
    11 & RX Error:\hspace{.1in}one or more CSCs violated 64-bit word boundary & bad event \\
    10 & DDU Clock-DLL Error:\hspace{.1in}the DDU lost it's clock for an unknown period of time; & may be OK \\
       & \hspace{.1in}some data may be lost & \\
     9 & DDU detected a CSC Error:\hspace{.1in}Timeout, DMB CRC, CFEB Sync/Overflow, or & bad event \\
       & \hspace{.1in}missing CFEB data & \\
     8 & DDU Lost In Event Error:\hspace{.1in}lost or mis-sequenced data was detected & bad event \Hline
     7 & DDU Lost In Data Error:\hspace{.1in}monitor logic hung by lost or mis-sequenced data & bad event, RESET \\
     6 & DDU Timeout Error:\hspace{.1in}CSC end of data not detected & bad event, RESET \\
     5 & DDU detected TMB or ALCT CRC Error & bad event \\
     4 & DDU Multiple Transmit Errors:\hspace{.1in}several RX or bit-errors occurred on a CSC  & bad event, RESET \\
     3 & DDU Sync Lost/Buffer Overflow Error & bad event, RESET \\
     2 & DDU detected Fiber Error:\hspace{.1in}bad fiber connection status & RESET \\
     1 & DDU detected DMB L1A Match Error:\hspace{.1in}bit-error or sync error & bad event \\
     0 & DDU detected CFEB CRC Error:\hspace{.1in}bit-error or wordcount error & bad event \Hline
  \end{tabular}
  \end{center}
\end{table}

The volume of DDU data will vary from event-to-event depending on the number
of CSCs with track stubs.  In the frequent case where there are no CSCs
with track stubs in a DDU, it will still send the DDU header and trailer
blocks (48 bytes).



\subsection{EMU DMB Format}
The Data Acquisition Motherboard
collects the data from every source on the CSC, including 4-5 CFEBs,
1 ALCT and 1 TMB.  When the L1A is received, each source signals % informs
the DMB if it has track-stub data for the event, then transmits
its 16-bit data to the DMB where it is buffered in a FIFO array.  
For those sources with data,
the DMB reads the event record from each source in sequence
(see Table~\ref{tab:EmuDMBformat}) and sends the data
to the DDU via 1.6 Gb/s fiber channel.
% In addition,
Additionally, %the DMB sends 
CSC buffer status, geometry and event information,
and CRC check words are contained
within the DMB header and trailer blocks.
% 64-bit table, in a landscape-rotated page
\begin{landscape}
\thispagestyle{empty}
\begin{table}[tbp]
  \caption{EMU DMB data format with track stub present.}\label{tab:EmuDMBformat}
    \begin{DDUtabular}{64}
      \bitNumEightByte
      \bitline{
        \ffield{4}{9h} & 
        \field{12}{DMBbx[11:0]} & 
        \field{4}{9h} & 
        \field{12}{Data\_available\_flags\_A} & 
        \field{4}{9h} & 
        \field{12}{DMB\_L1A[23:12]} & 
        \field{4}{9h} & 
        \field{12}{DMB\_L1A[11:0]} }
      \bitline{
	\ffield{4}{Ah} & 
	\field{4}{DMB\_sync} & 
	\field{8}{DMB\_L1A[7:0]} & 
	\field{4}{Ah} & 
	\field{5}{CFEB\_movlp} & 
	\field{7}{DMBbx[6:0]} & 
	\field{4}{Ah} & 
	\field{8}{DMB\_Crate} & 
	\field{4}{DMB\_ID} & 
	\field{4}{Ah} & 
	\field{12}{Data\_available\_flags\_B} }
      \bitline{
	\ffield{64}{ALCT\_Data\_Payload (see Table~\ref{tab:EmuALCTfull})} }
      \bitline{
	\ffield{64}{TMB\_Data\_Payload (see Tables~\ref{tab:EmuTMBshort} to \ref{tab:EmuTMBfull})} }
      \bitline{
	\ffield{64}{CFEB\_Data\_Payload (0-5 CFEBs, see Table~\ref{tab:EmuCFEBsca} and Section~\ref{sec:EmuCFEB})} }
%      \vdotline{64}
      \bitline{
	\ffield{4}{Fh} & 
	\field{12}{DMB\_timeout\_flags} & 
	\field{4}{Fh} & 
	\field{10}{DMB\_FIFO\_flags} & 
%        \parbox{3.4mm}{\centering\tiny DMB time} & 
	\tfield{2}{DMB time} & 
	\field{4}{Fh} & 
	\field{5}{CFEB\_movlp} & 
	\field{7}{DMB\_warn\_flags} & 
	\field{4}{Fh} & 
	\field{4}{DMBbx[3:0]} & 
        \field{8}{DMB\_L1A[7:0]} }
      \bitline{
	\ffield{4}{Eh} & 
	\field{1}{P} & 
	\field{11}{DMB\_CRC[21:11]} & 
	\field{4}{Eh} & 
	\field{1}{P} & 
	\field{11}{DMB\_CRC[10:0]} & 
	\field{4}{Eh} & 
	\field{8}{DMB\_Crate} & 
	\field{4}{DMB\_ID} & 
	\field{4}{Eh} & 
	\field{12}{FIFO\_flags} }
      \bitNumEightByte
    \end{DDUtabular}
\end{table}
P: \, DMB CRC Word parity flag.

\vspace{.5in}
\begin{table}[hbp]
  \caption{EMU DMB data format with no track stubs present.}\label{tab:EmuDMBempty}
    \begin{DDUtabular}{64}
      \bitNumEightByte
      \bitline{
        \ffield{4}{8h} & 
        \field{12}{DMBbx[11:0]} & 
        \field{4}{8h} & 
        \field{12}{000h} & 
        \field{4}{8h} & 
        \field{12}{DMB\_L1A[23:12]} & 
        \field{4}{8h} & 
        \field{12}{DMB\_L1A[11:0]} }
      \bitNumEightByte
    \end{DDUtabular}
\end{table}
\end{landscape}

The DMB headers and trailers carry several fields of interest for analysis:
\begin{description}
\item {\bf Data\_available\_flags\_(A/B)}: These words tell which CSC sources
have track-stub data available (DAV); some bits are repeated for voting at
the DDU level.
\begin{quotation}
  A = TMB\_DAV(1) + ALCT\_DAV(1) + CFEB\_CLCT\_SENT(5:1) + CFEB\_DAV(5:1)

  B = TMB\_DAV(1) + CLCT-DAV-Mismatch(1) + ALCT\_DAV(1) + CLCT-DAV-Mismatch(1)

\hspace{.2in} + TMB\_DAV(1) + CLCT-DAV-Mismatch(1) + ALCT\_DAV(1) + CFEB\_DAV(5:1)
\end{quotation}
\item {\bf DMB\_L1A}: equivalent to LV1\_id in the CMS Common Data Format.
\item {\bf DMBbx}: equivalent to the BX\_id.
\item {\bf DMB\_Crate $+$ DMB\_ID}: these identify which CSC the data comes
from.
\item {\bf DMB\_timeout\_flags}: identifies CSC sources missing the Start
or End of data as indicated below (12-bits).
\begin{quotation}
 CFEB\_End\_Timeout(5:1) + ALCT\_End\_Timeout(1) + TMB\_End\_Timeout(1)

\hspace{.2in} + CFEB\_Start\_Timeout(5:1)
\end{quotation}
\item {\bf DMB time}: two additional timeout flags; \, ALCT\_Start\_Timeout(1) + TMB\_Start\_Timeout(1)
\end{description}

The volume of DMB data will vary from event-to-event depending on the number
of sources with track-stub data.  In the frequent case where there are no
sources with track-stub data in the CSC (empty CSC condition), the DMB
will send a special 64-bit word (see Table~\ref{tab:EmuDMBempty})
with the minimal information required to maintain
synchronization with the DDU.
This information is checked and summarized at the DDU, then the
empty CSC words are excluded from the data stream by zero-suppression logic.

\subsection{EMU CFEB Format}\label{sec:EmuCFEB}
The Cathode Front-End Boards
send 16-bit data words in multiples of 4 (as do all CSC sources),
and the 16th bit is reserved as a special control word flag which is set to
zero for all normal data words.  Each CFEB has three possible responses to
an L1A:
\begin{enumerate}
\item   Send nothing: this CFEB has no track-stub data.
\item   Send the Data Available signal, followed by SCA time-sample data.
\item   As above, but some time samples are skipped due to temporary
``SCA Full'' condition.
\end{enumerate}

For the second case (normal data transmission), each CFEB data stream
is composed of 100 16-bit words per time sample.  This includes 96 SCA data
words for the 12-bit digitized cathode strip data, plus 4 additional
words at the end of each time sample: a CRC word, 2 status words and
an end marker (not guaranteed unique).
% making the word count an integer multiple of 4.
Therefore, the full 16-bit word count per CFEB (CF$_{WC}$) depends on the
number of time samples (N$_{ts}$) digitized:

% \begin{displaymath}
\begin{quote}
CF$_{WC}$ = \{[ (16 strips) * (6 layers)] 
       + 1 CRC + 2 CFEB\_Info + 1 End\_marker\} * N$_{ts}$
\end{quote}
% \end{displaymath}

N$_{ts}$ (number of time samples) is defined in the firmware
or through slow control,
typically set to 8 for LHC running (may be 16 for calibration runs)
yielding CF$_{WC}$(typical)=800.  However, in case of a temporary
SCA Full condition
the CF$_{WC}$ may deviate from this scheme (case 3 above),
and any % missing/lost/absent/skipped
time samples with no SCA data have their 100 data words replaced
by a 16-bit ``B-code'' word repeated 4 times:
\begin{quote}
SCA\_Full word format:  B2h + Block\_number(4) + FIFO\_word\_count(4)
\end{quote}
SCA Full is a temporary condition that can only occur during an unusually
active burst of triggers; the CFEB will automatically recover from
this and no Resets are required.

The 96 SCA data words in every time sample follow the format
in Table~\ref{tab:EmuCFEBsca}, where the definitions for the
last 4 CFEB words are also found.

\begin{table}[htbp]
  \caption{EMU CFEB-SCA data format.}\label{tab:EmuCFEBsca}
  \begin{center}
  \begin{tabular}{|c|l|}
    \hline
    CFEB Data Bit   &   SCA Data Word Bit Definition \HHline
    15 & Always LOW for data words, HIGH for SCA\_Full words. \\
    14 & Overlap sample flag, normally HIGH; set LOW when two LCTs share a time sample. \\
    13 & Serialized 16-bit CFEB-SCA controller status (see below). \\
    12 & Out of range flag from CFEB ADC. \\
    11-0 & 12-bit Digitized SCA ADC data. \Hline
  \end{tabular}
  \end{center}
\begin{quotation}
Word 97:  CRC\_Word(15) created using a CRC-15 algorithm.

Word 98:  7h + L1PIPE\_EMPTY(1) + LCTPIPE\_EMPTY(1) + L1PIPE\_FULL(1) 

\hspace{.5in} + LCTPIPE\_FULL(1) + LCTPIPE\_CNT(4) + NF\_SCA(4)

Word 99:  7h + DMB\_L1A(6) + L1PIPE\_CNT(5) + L1PIPE\_WARN(1)

Word 100:  7FFFh (end of sample marker).
\end{quotation}
\end{table}
L1PIPE and LCTPIPE refer to CFEB-SCA Controller internal pipeline status;
NF\_SCA is the number of free SCA blocks (12 max).

The relation between CSC geometry and the data words from a single CFEB is described by the following series of nested loops:
\begin{quote}
do (N$_{ts}$  time samples)\{
\begin{quotation}
do (Gray code loop over 16 CSC Strips; S=0,1,3,2,6,7,5,4,12,13,15,14,10,11,9,8)\{
\begin{quotation}
do (loop over 6 CSC Layers; L=3,1,5,6,4,2)\{

\hspace{.2in} SCA data word

\}
\end{quotation}

\}

CRC Word

CFEB Word 98

CFEB Word 99

CFEB Word 100
\end{quotation}
\}
\end{quote}

Bit 13 of each SCA data word carries the serialized 16-bit CFEB-SCA
Controller status word, containing trigger and SCA information.
This data word is serialized (LSB first) with one bit in each of
the 16 strips read out, yielding the 16-bit word.  Due to the
innermost-loop over the 6 CSC layers, each bit of a 16-bit
Controller status word is actually sent 6 times in a
row (once for each layer).  Since there are 16 strips read out
for each time sample, the complete Controller status word can be
reconstructed independently in every time sample.  The 16-bit word
is defined as follows (highest-to-lowest bit):
\begin{quotation}
Controller status:  TS\_FLAG(1) + FULL\_SCA(1) + LCT\_PHASE(1) 
+ L1A\_PHASE(1)

\hspace{.6in} + SCA\_BLK(4) + TRIG\_TIME(8)
\end{quotation}
TRIG\_TIME indicates 
the starting capacitor in the 8-capacitor SCA block
%which of the eight time samples in the 400ns SCA block
(lowest bit is the first capacitor, highest bit the eighth capacitor).
%corresponds to the arrival of the LCT; it should be a fixed phase
%relative to the peak of the CSC pulse.
SCA\_BLK is the SCA Capacitor block used
for this time sample. L1A\_PHASE and LCT\_PHASE show the phase
of the 50ns CFEB digitization clock at the time the trigger
was received (1=clock high, 0=clock low).  FULL\_SCA indicates
there were no free SCA blocks left at digitization time.
The TS\_FLAG bit indicates the number of time samples to digitize
per event; high=16 time samples, low=8 time samples.

\subsection{EMU TMB Format}
% What does it do?  Combines Anode & Cathode trigger hits into muon track
% stubs with pattern matching algorithms and passes this to MPC & L1 system.

The Trigger Motherboard receives raw cathode hit information from the
CFEBs and searches for high-p$_t$ patterns.  It combines the cathode
result with anode patterns received from the ALCT board to identify
muon track stubs in the CSC.  The muon track information is then sent
to the Level-1 trigger system.  If an L1A is issued for a muon track,
the TMB sends the cathode hit data to the DMB; % and the
anode hit data from the ALCT is passed to the DMB via the TMB.

The TMB has several options for data format that can be
selected based on data rate considerations.  The shortest is the
TMB Short Header Only format, which is shown in Table~\ref{tab:EmuTMBshort}.
In this mode there is no raw cathode trigger hit (Triad) information sent.

The longest format is the TMB Full Readout option (Table~\ref{tab:EmuTMBfull}),
which contains the cathode processing
results as well as the raw cathode trigger Triad data.
Intermediate options include Local Triad Readout, where only those
Triads near the track stub are read out, and
TMB Full Header Only, which is the same
as the Full Readout with none of the Triad data.
In all of these cases there are 2 optional words available
near the end to maintain 4-word multiples as needed.

\begin{table}[tbp]
  \caption{EMU TMB data format, Short Header Only mode.}\label{tab:EmuTMBshort}
    \begin{TMBtabular}{16}
      Word \# & \bitNumTwoByte\hline
      \bitline{0 & 
        \field{1}{0} & 
        \field{3}{6h} & 
        \field{12}{B0Ch} }
      \bitline{1 & 
        \field{1}{0} & 
        \field{3}{FIFO\_Mode$^{*1}$} & 
        \field{7}{CFEBs\_in\_Readout} & 
        \field{5}{Time\_bins} }
      \bitline{2 & 
        \field{1}{0} & 
        \field{2}{L1\_type$^{*2}$} & 
        \field{5}{Board\_ID} & 
        \field{4}{CSC\_ID} & 
        \field{4}{TMB\_L1A[3:0]} }
      \bitline{3 & 
        \field{1}{0} & 
        \field{1}{x} & 
        \field{2}{r\_type$^{*3}$} & 
        \field{12}{TMBbx[11:0]} }
      \bitline{4 & 
        \field{4}{Dh} & 
        \field{1}{1} & 
        \field{11}{CRC22[10:0]} }
      \bitline{5 & 
        \field{4}{Dh} & 
        \field{1}{1} & 
        \field{11}{CRC22[21:11]} }
      \bitline{6 & 
        \field{4}{Dh} & 
        \field{12}{EEFh} }
      \bitline{7 & 
        \field{4}{Dh} & 
        \field{1}{1} & 
        \field{11}{TMB\_word\_count[10:0]} }
    \end{TMBtabular}

$^{*1}$ \, See Table~\ref{tab:EmuTMBfmode}.

$^{*2}$ \, See Table~\ref{tab:EmuTMBl1type}.

$^{*3}$ \, See Table~\ref{tab:EmuTMBrtype}.
\end{table}
%\vspace{.5in}
\begin{table}[htbp]
  \caption{FIFO Mode Definitions.}\label{tab:EmuTMBfmode}
  \begin{center}
    \begin{tabular}{|c||c|l|}
      \hline
      FIFO\_Mode & Raw Trigger Hits & Header Size \Hline
      0 & None  & Full \\
      1 & Full  & Full \\
      2 & Local & Full \\
      3 & None  & Short \\
      4 & None  & None \Hline
    \end{tabular}
  \end{center}
\end{table}
\begin{table}[htbp]
  \caption{L1 Type Definitions.}\label{tab:EmuTMBl1type}
  \begin{center}
    \begin{tabular}{|c||l|}
      \hline
      L1\_type  & Comment \Hline
      0 & Normal CLCT trigger with cathode data and L1A match\\
      1 & ALCT only trigger, no cathode data and no readout\\
      2 & L1A only, no cathode or anode data and no readout\\
      3 & LCT triggered, but no L1A match and no readout\Hline
    \end{tabular}
  \end{center}
\end{table}
\begin{table}[htbp]
  \caption{Record Type Definitions.}\label{tab:EmuTMBrtype}
  \begin{center}
    \begin{tabular}{|c||c|l|}
      \hline
      r\_type & Raw Trigger Hits & Header Size \Hline
      0 & None  & Full \\
      1 & Full  & Full \\
      2 & Local & Full \\
      3 & None  & Short (no buffer available at pre-trigger)\Hline
    \end{tabular}
  \end{center}
\end{table}

\begin{table}[htbp]
  \caption{EMU TMB data format, full readout mode.}\label{tab:EmuTMBfull}
    \begin{TMBtabular}{16}
      Word \# & \bitNumTwoByte\hline
      \bitline{0 & 
        \field{1}{0} & 
        \field{3}{6h} & 
        \field{12}{B0Ch} }
      \bitline{1 & 
        \field{1}{0} & 
        \field{3}{FIFO\_Mode} & 
        \field{7}{CFEBs\_in\_Readout} & 
        \field{5}{Time\_bins} }
      \bitline{2 & 
        \field{1}{0} & 
        \field{2}{L1\_type} & 
        \field{5}{Board\_ID} & 
        \field{4}{CSC\_ID} & 
        \field{4}{TMB\_L1A[3:0]} }
      \bitline{3 & 
        \field{1}{0} & 
        \field{1}{x} & 
        \field{2}{r\_type} & 
        \field{12}{TMBbx[11:0]} }
      \bitline{4 & 
        \field{1}{0} & 
        \field{1}{x} & 
        \field{5}{Tbin\_before\_pretrig} & 
        \parbox{3.4mm}{\centering\tiny has buf} & 
        \field{3}{\# CFEBs} & 
        \field{5}{\# Header\_Frames} }
      \bitline{5 & 
        \field{1}{0} & 
        \field{2}{xx} & 
        \parbox{3.4mm}{\centering\tiny has pretrg} & 
        \field{8}{Trig\_source\_vector} & 
        \field{4}{L1A\_Tx} }
      \bitline{6 & 
        \field{1}{0} & 
        \field{1}{x} & 
        \field{4}{Run\_ID} & 
        \field{5}{CFEBs\_instantiated} & 
        \field{5}{CFEB\_LCTs} }
      \bitline{7 & 
        \field{1}{0} & 
        \field{2}{xx} & 
        \parbox{3.4mm}{\centering\tiny Sync Err} & 
        \field{12}{BX\_at\_LCT[11:0]} }
      \bitline{8 & 
        \field{1}{0} & 
        \field{15}{Cathode\_LCT0[14:0]} }
      \bitline{9 & 
        \field{1}{0} & 
        \field{15}{Cathode\_LCT1[14:0]} }
      \bitline{10 & 
        \field{1}{0} & 
        \field{2}{xx} & 
        \parbox{3.4mm}{\centering\tiny invalid patt.} & 
        \field{6}{Cathode\_LCT1[20:15]} & 
        \field{6}{Cathode\_LCT0[20:15]} }
      \bitline{11 & 
        \field{1}{0} & 
        \field{4}{Triad\_persistence} & 
        \field{4}{ALCT\_match\_time} & 
        \field{2}{$\Delta$BX\_LCT1} & 
        \field{2}{$\Delta$BX\_LCT0} & 
        \parbox{3.4mm}{\centering\tiny CLCT Only} & 
        \parbox{3.4mm}{\centering\tiny ALCT Only} & 
        \parbox{3.4mm}{\centering\tiny TMB Match} }
      \bitline{12 & 
        \field{1}{0} & 
        \field{15}{MPC\_muon0\_frame0[14:0]} }
      \bitline{13 & 
        \field{1}{0} & 
        \field{15}{MPC\_muon0\_frame1[14:0]} }
      \bitline{14 & 
        \field{1}{0} & 
        \field{15}{MPC\_muon1\_frame0[14:0]} }
      \bitline{15 & 
        \field{1}{0} & 
        \field{15}{MPC\_muon1\_frame1[14:0]} }
      \bitline{16 & 
        \field{1}{0} & 
        \field{1}{x} & 
        \field{3}{ds\_thresh} & 
        \field{3}{hs\_thresh} & 
        \field{2}{xx} & 
        \field{2}{MPC\_accept} & 
        \field{4}{Muon\_Frame\_MSBs} }
      \bitline{17 & 
        \field{1}{0} & 
        \field{1}{x} & 
        \field{14}{Buffer\_write\_flags} }
      \bitline{18 & 
        \field{1}{0} & 
        \field{2}{xx} & 
        \field{13}{Buffers\_busy} }
      \bitline{19 & 
        \field{1}{0} & 
        \field{1}{x} & 
        \field{14}{Buffer\_read\_flags} }
      \bitline{20 & 
        \field{1}{0} & 
        \field{3}{Timeout} & 
        \field{12}{Discard\_flags} }
      \bitline{21 & 
        \field{1}{0} & 
        \field{1}{x} & 
        \field{14}{Firmware\_revision\_code} }
      \bitline{22 & 
        \field{1}{0} & 
        \field{3}{6h} & 
        \field{12}{E0Bh} }
      \bitline{ & 
        \field{1}{0} & 
        \field{3}{CFEB\# (0-4)} & 
        \field{4}{Tbin\# (0-n, n$<$7)} & 
        \field{8}{Triad data, 8-bits per layer (0-5)} }
%        \field{15}{Triad loop for 8-bit data:  CFEB 0-4, Tbin 0-n, layer (0-5)}}
      \vdotline{16}
      \bitline{ & 
        \field{1}{0} & 
        \field{3}{6h} & 
        \field{12}{E0Ch} }
      \bitline{ & 
        \field{1}{0} & 
        \field{3}{0h} & 
        \field{12}{AAAh (as needed to maintain 64-bit boundary} }
      \bitline{ & 
        \field{1}{0} & 
        \field{3}{0h} & 
        \field{12}{555h (as needed to maintain 64-bit boundary} }
      \bitline{ N-3 & 
        \field{4}{Dh} & 
        \field{1}{1} & 
        \field{11}{CRC22[10:0]} }
      \bitline{ N-2 & 
        \field{4}{Dh} & 
        \field{1}{1} & 
        \field{11}{CRC22[21:11]} }
      \bitline{ N-1 & 
        \field{4}{Dh} & 
        \field{12}{E0Fh} }
      \bitline{ N & 
        \field{4}{Dh} & 
        \field{1}{1} & 
        \field{11}{TMB\_word\_count[10:0]} }
    \end{TMBtabular}
\end{table}



\subsection{EMU ALCT Format}
Currently the ALCT has only the Full Readout data format option
(Table~\ref{tab:EmuALCTfull}), which reads out the wire group hit
data for all LCT chips.
It is expected that ALCT will
add a smaller Local Readout option in the future,
% This option will only those
such that only wire group hits near the track stub are read out.
\begin{table}[tbp]
  \caption{EMU ALCT data format, full readout.}\label{tab:EmuALCTfull}
    \begin{TMBtabular}{16}
      Word \# & \bitNumTwoByte\hline
      \bitline{0 & 
        \field{1}{0} & 
        \field{3}{6h} & 
        \field{1}{0} & 
        \field{3}{Board\_ID} & 
        \field{4}{CSC\_ID} & 
        \field{4}{ALCT\_L1A[3:0]} }
      \bitline{1 & 
        \field{1}{0} & 
        \field{1}{0} & 
        \field{3}{xxx} & 
        \field{1}{2nd} & 
        \field{1}{1st} & 
        \field{1}{Ext} & 
        \field{1}{L1A} & 
        \field{5}{Time\_bins} & 
        \field{2}{FIFO\_Mode} }
      \bitline{2 & 
        \field{1}{0} & 
        \field{1}{0} & 
        \field{2}{xx} & 
        \field{12}{ALCTbx[11:0]} }
      \bitline{3 & 
        \field{1}{0} & 
        \field{1}{0} & 
        \field{7}{ActiveLCTchips} & 
        \field{7}{LCTchips\_read\_out} }
      \bitline{4 & 
        \field{1}{0} & 
        \field{15}{LCT0[14:0]} }
      \bitline{5 & 
        \field{1}{0} & 
        \field{15}{LCT0[29:14]} }
      \bitline{6 & 
        \field{1}{0} & 
        \field{15}{LCT1[14:0]} }
      \bitline{7 & 
        \field{1}{0} & 
        \field{15}{LCT1[29:14]} }

      \bitline{ & 
        \field{1}{0} & 
        \field{3}{LCTchip\# (0-6)} & 
        \field{4}{Tbin\# (0-n, n$<$5)} & 
        \field{8}{Wire Group data, 2 8-bit words per layer (0-5)} }
%        \field{15}{Wiregroup loop for 8-bit data:  CFEB 0-4, Tbin 0-n, layer (0-5)}}
      \vdotline{16}
      \bitline{ N-3 & 
        \field{4}{Dh} & 
        \field{1}{0} & 
        \field{11}{CRC22[10:0]} }
      \bitline{ N-2 & 
        \field{4}{Dh} & 
        \field{1}{0} & 
        \field{11}{CRC22[21:11]} }
      \bitline{ N-1 & 
        \field{4}{Dh} & 
        \field{12}{E0Dh ``Evener'' word} }
      \bitline{ N & 
        \field{4}{Dh} & 
        \field{2}{0} & 
        \field{10}{ALCT\_word\_count[9:0]} }
    \end{TMBtabular}
\end{table}


\subsection{EMU CSC Trackfinder}



% Anode & Cathode hit information from the front end boards (AFEB & CFEB),
% trigger primitives from TMB and ALCT, CSC 
% DMB headers carry geometry and event information for the CSC.

%%% Local Variables: 
%%% mode: latex
%%% TeX-master: "DataFormats"
%%% End: 
