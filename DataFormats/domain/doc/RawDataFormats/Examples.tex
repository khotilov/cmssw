\clearpage
\section*{Example of formatting}

Here are some examples of how to create tables with data formats.

Table~\ref{tab:test1} shows an example of a 16-bit table.
% Example of a 16-bit table
\begin{table}[htb]
  \caption{Example of a table for a 16-bit format}\label{tab:test1}
    \begin{bittabular}{16}
      & \bitNumTwoByte
      \bitline{Header 1 & 1 & 0 & 0 & 1 & \field{12}{DMB LA1(11:0)}}
      \bitline{Header 1 & 1 & 0 & 0 & 1 & \field{12}{DMB LA1(23:12)}}
      \bitline{Header 1 & 1 & 0 & 0 & 1 & a & b & \field{5}{CFEB\_ACTIVE} & \field{5}{CFEB\_DAV}}
      \bitline{Header 1 & 1 & 0 & 0 & 1 & \field{12}{DMB BXN}}
      & \bitNumTwoByte
    \end{bittabular}
\end{table}

Table~\ref{tab:test2} shows an example of a 32-bit table.
% Example of a 32-bit table
\begin{table}[htb]
  \caption{Example of a table for a 32-bit format}\label{tab:test2}
    \begin{bittabular}{32}
      & \bitNumFourByte
      \bitline{TDC & 0 & 0 & 0 & 0 & \field{4}{TDC ID} & \field{12}{Event ID} & \field{12}{Bunch ID}}
      \bitline{ROS & 0 & 0 & 0 & \field{5}{ROB ID (0-30)} & \field{12}{Event ID} & \field{12}{Bunch ID}}
      & \bitNumFourByte
    \end{bittabular}
\end{table}

%$\leftarrow$\hfill Width of a line \hfill$\rightarrow$

Table~\ref{tab:test3} shows an example of a 64-bit table in a separate
landscape-oriented page.

% Example of 64-bit table, in a landscape-rotated page
\begin{landscape}
\begin{table}[htb]
  \caption{Example of a table for a 64-bit format}\label{tab:test3}
    \begin{bittabular}{64}
      & \bitNumEightByte
      \bitline{K & 
        \field{4}{BOE\_1} & 
        \field{4}{Evt\_ty} &
        \field{24}{LV1\_id} & 
        \field{12}{BX\_id} & 
        \field{12}{Source\_id} &
        \field{4}{FOV} &
        H & x & \$ & \$}
      \bitline{ D & \field{64}{Sub-detector payload} }
      \bitline{ D & \field{64}{Sub-detector payload} }
      \bitline{K & 
        \field{4}{EOE\_1} & 
        \field{4}{xxxx} &
        \field{24}{Evt\_lgth} & 
        \field{16}{CRC} & 
        \field{4}{xxxx} &
        \field{4}{EVT\_stat} &
        \field{4}{TTS} & T & x & \$ & \$}
      & \bitNumEightByte
    \end{bittabular}
\end{table}

%$\leftarrow$\hfill Width of a line \hfill$\rightarrow$

Legend: 
\begin{itemize}
\item BOE\_n: Identifier for the beginning of an event fragment (BEO\_1 = hex 5)
\item Evt\_ty: Event type identifier (see notes below)
\item LV1\_id: The level-1 event number generated by the TTC system. The first event after a TTC reset is tagged with no. 1
\item BX\_id: The bunch crossing number. Reset on every LHC orbit
\item Source\_id: Unambiguously identify the data source (FED/DCC): 2 bits are reserved for FED internal usage
\item FOV: Version identifier of the FED data format
\item H: when set to '0', the current header word is the last one. When set to '1', another header word is following.
\item EOE\_n: Identifier for the end of an event fragment (EOE\_1 = hex A)
\item Evt\_lgth: The length of the event fragment counted in 64-bit words including header and trailer
\item CRC: Cyclic Redundancy Code of the event fragment including header and trailer (see notes below)
\item Evt\_stat: Event fragment status information (see notes below)
\item TTS: Current values of the TTS bits
\item T: when set to '0', the current trailer word is the last one. When set to '1', another trailer word is following.
\item x: Indicates a reserved bit
\item \$: Indicates a bit used by the S-LINK64 hardware
\end{itemize}

\end{landscape}




%%% Local Variables: 
%%% mode: latex
%%% TeX-master: "DataFormats"
%%% End: 
