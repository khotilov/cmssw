%======================================================================
% 
% For instructions see the file README.txt.
%
% Author: N. Amapane  
% $Date: 2005/07/12 09:44:41 $  $Revision: 1.9 $
%
%======================================================================

\documentclass[a4paper]{cmspaper}
\usepackage{lineno} % line numbering for draft
\usepackage{ifthen}
%\usepackage{multido}
\usepackage{array}

% Temporary way to set the input path (until a configuration for
% CMSSW is ready)
\makeatletter 
\def\input@path{{./../../../}} 
\makeatother

%======================================================================
% Check for PDFLaTeX/LaTeX 

\newif\ifpdf
\ifx\pdfoutput\undefined
\pdffalse 
\else
\pdfoutput=1 
\pdftrue
\fi

\ifpdf  % ============== we are running PDFLaTeX
\usepackage{color}
\usepackage[pdftex]{graphicx,graphics}
\usepackage{epsfig}
\usepackage{pdflscape}
\usepackage[bookmarksnumbered,bookmarksopen,bookmarksopenlevel=1,
              colorlinks,
              linkcolor=blue,                      
              citecolor=blue, 
              urlcolor=blue]{hyperref}                                 
\pdfinfo{
  /Title  (Description of Detector Raw Data Formats)
  /Author (...)
  /Keywords (FED Raw data Formats)
  }

\DeclareGraphicsExtensions{.pdf}

\else   % ============== we are not running PDFLaTeX
\usepackage{epsfig}
\usepackage{lscape}

\special{!userdict begin /bop-hook{gsave 200 30 translate
         65 rotate /Times-Roman findfont 216 scalefont setfont
         0 0 moveto 0.93 setgray (DRAFT) show grestore} def end}

\DeclareGraphicsExtensions{.eps}

%fake pdflatex commands.
\newcommand{\pdfbookmark}[3][1]{}
\newcommand{\href}[2]{#2}

\fi

%==============================================================================

% A set of definitions to create tables with data formats.

% The header line with byte numbers
\newcounter{bitCounter}
\def\bitNum{\addtocounter{bitCounter}{-1} \arabic{bitCounter}}
\def\oneByte{\bitNum&\bitNum&\bitNum&\bitNum&\bitNum&\bitNum&\bitNum&\bitNum}

% Row with byte numbers for a 8-bit fragment
\newcommand{\bitNumOneByte}{
  \setcounter{bitCounter}{8}
  \oneByte\\\hline 
}

% Row with byte numbers for a 16-bit fragment
\newcommand{\bitNumTwoByte}{
  \setcounter{bitCounter}{16}
  \oneByte&\oneByte\\\hline 
}

% Row with byte numbers for a 32-bit fragment
\newcommand{\bitNumFourByte}{
  \setcounter{bitCounter}{32}
  \oneByte&\oneByte&\oneByte&\oneByte\\\hline 
}

% Row with byte numbers for a 64-bit fragment
\newcommand{\bitNumEightByte}{
  \setcounter{bitCounter}{64}
  \oneByte&\oneByte&\oneByte&\oneByte&
  \oneByte&\oneByte&\oneByte&\oneByte\\\hline 
}

% A field spanning several bits
\newcommand{\field}[2]{\multicolumn{#1}{c|}{ #2}}

% A row (simple shortcut)
\newcommand{\bitline}[1]{#1\\\hline}

% A tabular with the specified number of bits (plus one leading description field)
\newenvironment{bittabular}[2][3.5mm]
{\hspace{-0.1\linewidth}\begin{minipage}[t]{1.2\linewidth}
\begin{center}
\small\tabcolsep0.2mm
\begin{tabular}{|c|*{#2}{m{#1}<{\centering}|}}\hline}
{\end{tabular}\end{center}\end{minipage}}

%======================================================================

\begin{document}
\begin{titlepage}

  \internalnote{2005/XXX}
  \date{\today}

  \title{Description of Detector Raw Data Formats}
  \begin{Authlist}
    N.~Amapane\Aref{1},
    R.~Alemany Fernandez\Aref{2},
    D.~Kotlinski\Aref{3},
    G.~Bruno\Aref{4},
    M.~Maggi\Aref{5},
    J.~Mans\Aref{6},
    E.~Meschi\Aref{1}

    \Anotfoot{1}{CERN, Geneva, Switzerland}
    \Anotfoot{2}{LIP, Lisboa, Portugal}
    \Anotfoot{3}{Paul Scherrer Inst., Villigen, Switzerland}
    \Anotfoot{4}{Univ. Catholique de Louvain, Louvain-la-Neuve, Belgium}
    \Anotfoot{5}{Univ. di Bari e Sez. dell'INFN, Bari, Italy}
    \Anotfoot{6}{University of Maryland, USA}

  \end{Authlist}

  \begin{abstract}
    \pdfbookmark[1]{Abstract}{Abstract}
    This note describes the format of the raw data sent to the DAQ by
    the FEDs of the various detectors.
  \end{abstract} 
  
\end{titlepage}

\setcounter{page}{2}

\linenumbers % line numbering for draft

\section{Introduction}\label{sec:Introduction}

Some introduction; outline of the note

Here are some examples of tables with data formats:

% Example of a table for a 32-bit format
\begin{table}[htb]
  \caption{Example of a table for a 32-bit format}\label{tab:test1}
    \begin{bittabular}{32}
      & \bitNumFourByte
      \bitline{TDC & 0 & 0 & 0 & 0 & \field{4}{TDC ID} & \field{12}{Event ID} & \field{12}{Bunch ID}}
      \bitline{ROS & 0 & 0 & 0 & \field{5}{ROB ID (0-30)} & \field{12}{Event ID} & \field{12}{Bunch ID}}
      & \bitNumFourByte
    \end{bittabular}
\end{table}

$\leftarrow$\hfill Width of a line \hfill$\rightarrow$

% Example of a table for a 64-bit format, in a landscape-rotated page
\begin{landscape}
\begin{table}[htb]
  \caption{Example of a table for a 64-bit format}\label{tab:test2}
    \begin{bittabular}{64}
      & \bitNumEightByte
      \bitline{K & 
        \field{4}{BOE\_1} & 
        \field{4}{Evt\_ty} &
        \field{24}{LV1\_id} & 
        \field{12}{BX\_id} & 
        \field{12}{Source\_id} &
        \field{4}{FOV} &
        H & x & \$ & \$}
      \bitline{ D & \field{64}{Sub-detector payload} }
      \bitline{ D & \field{64}{Sub-detector payload} }
      \bitline{K & 
        \field{4}{EOE\_1} & 
        \field{4}{xxxx} &
        \field{24}{Evt\_lgth} & 
        \field{16}{CRC} & 
        \field{4}{xxxx} &
        \field{4}{EVT\_stat} &
        \field{4}{TTS} & T & x & \$ & \$}
      & \bitNumEightByte
    \end{bittabular}
\end{table}

$\leftarrow$\hfill Width of a line \hfill$\rightarrow$

Legend: 
\begin{itemize}
\item BOE\_n: Identifier for the beginning of an event fragment (BEO\_1 = hex 5)
\item Evt\_ty: Event type identifier (see notes below)
\item LV1\_id: The level-1 event number generated by the TTC system. The first event after a TTC reset is tagged with no. 1
\item BX\_id: The bunch crossing number. Reset on every LHC orbit
\item Source\_id: Unambiguously identify the data source (FED/DCC): 2 bits are reserved for FED internal usage
\item FOV: Version identifier of the FED data format
\item H: when set to '0', the current header word is the last one. When set to '1', another header word is following.
\item EOE\_n: Identifier for the end of an event fragment (EOE\_1 = hex A)
\item Evt\_lgth: The length of the event fragment counted in 64-bit words including header and trailer
\item CRC: Cyclic Redundancy Code of the event fragment including header and trailer (see notes below)
\item Evt\_stat: Event fragment status information (see notes below)
\item TTS: Current values of the TTS bits
\item T: when set to '0', the current trailer word is the last one. When set to '1', another trailer word is following.
\item x: Indicates a reserved bit
\item \$: Indicates a bit used by the S-LINK64 hardware
\end{itemize}

\end{landscape}




%%% Local Variables: 
%%% mode: latex
%%% TeX-master: "DataFormats"
%%% End: 
 
\section{Common Data Format}\label{sec:CommonFormat}

When an event fragment is ready to be transmitted to the DAQ, the FED
encapsulates the data according to the common CMS data format (see
below) and writes the data into the S-LINK64 port. The encapsulation
words are control words flagged by the S-LINK64 control bit. Detector
payload are data words.




%%% Local Variables: 
%%% mode: latex
%%% TeX-master: "DataFormats"
%%% End: 
 
\section{Silicon Strip Data Format}\label{sec:SiStrip}
The silicon strip tracker Front End Driver readout board will be able to operate in
various modes. The modes enable running for different types of physics events, running
for calibration or commissioning. The modes are [ref????]:
\begin{itemize}
\item Virgin Raw Data mode - in this mode the FED performs no data processing at all. This mode
is intended for commissioning and monitoring running.
\item Processed Raw Data mode - in this mode the FED performs pedestal subtraction and re-ordering (where
the strip data are re-ordered out of APV-MUX order into strip order). This mode is intended for use
when colliding heavy ions.
\item Zero Suppressed mode - in this mode the FED performs pedestal subtraction, re-ordering, common mode noise
calculation and subtraction, and zero suppression. This is intended to be the 'normal' running mode for 
proton-proton collisions.
\item Scope mode - in this mode the FED does not perform any APV header finding, and simply captures the data
entering a channel up to a pre-defined number of bytes. This enables the monitoring of APV tick marks to 
allow for tasks such as syncronization of FED channels.
\end{itemize}

In each mode the FED outputs the data with a DAQ header, a tracker specific header and a DAQ trailer. The DAQ
header was defined in the DAQ TDR [ref????] and consists of event type, level 1 trigger number, bunch crossing number,
source ID, etc. The tracker specific header contains information needed to check tracker and FED operation. 
There are two different tracker specific headers [ref???]: 
\begin{itemize}
\item Full Debug mode - this header contains APV error flags, FED status registers, data lengths, etc. This 
header format willbe used for commissioning, syncronization and raw data mode running.
\item APV Error mode - this will be the 'standard' running mode during a physics run (when the FED is operating
in zero suppressed mode). It is designed to be as small as possible, and so only contains as much information as
is needed for event reconstruction and monitoring.
\end{itemize}

\subsubsection {Silicon Strip FED Event Size}


The event buffer produced by the FED in raw data and processed raw data mode has a fixed size of 49,624 bytes 
(for full debug mode header) or 49,520 bytes (for APV error mode header). In scope mode the event size depends
on the number of bytes captured per FED channel. The maximum allowed is 512 bytes per channel, which results in
an event size maximum the same as raw data mode (depending on header format). In zero suppression mode the 
event size depends on the occupancy in the tracker, and the zero suppression thresholds.


%%% Local Variables: 
%%% mode: latex
%%% TeX-master: "DataFormats"
%%% End: 

\section{Silicon Strip Data Format}\label{sec:SiStrip}
The silicon strip tracker Front End Driver readout board will be able to operate in
various modes. The modes enable running for different types of physics events, running
for calibration or commissioning. The modes are [ref????]:
\begin{itemize}
\item Virgin Raw Data mode - in this mode the FED performs no data processing at all. This mode
is intended for commissioning and monitoring running.
\item Processed Raw Data mode - in this mode the FED performs pedestal subtraction and re-ordering (where
the strip data are re-ordered out of APV-MUX order into strip order). This mode is intended for use
when colliding heavy ions.
\item Zero Suppressed mode - in this mode the FED performs pedestal subtraction, re-ordering, common mode noise
calculation and subtraction, and zero suppression. This is intended to be the 'normal' running mode for 
proton-proton collisions.
\item Scope mode - in this mode the FED does not perform any APV header finding, and simply captures the data
entering a channel up to a pre-defined number of bytes. This enables the monitoring of APV tick marks to 
allow for tasks such as syncronization of FED channels.
\end{itemize}

In each mode the FED outputs the data with a DAQ header, a tracker specific header and a DAQ trailer. The DAQ
header was defined in the DAQ TDR [ref????] and consists of event type, level 1 trigger number, bunch crossing number,
source ID, etc. The tracker specific header contains information needed to check tracker and FED operation. 
There are two different tracker specific headers [ref???]: 
\begin{itemize}
\item Full Debug mode - this header contains APV error flags, FED status registers, data lengths, etc. This 
header format willbe used for commissioning, syncronization and raw data mode running.
\item APV Error mode - this will be the 'standard' running mode during a physics run (when the FED is operating
in zero suppressed mode). It is designed to be as small as possible, and so only contains as much information as
is needed for event reconstruction and monitoring.
\end{itemize}

\subsubsection {Silicon Strip FED Event Size}


The event buffer produced by the FED in raw data and processed raw data mode has a fixed size of 49,624 bytes 
(for full debug mode header) or 49,520 bytes (for APV error mode header). In scope mode the event size depends
on the number of bytes captured per FED channel. The maximum allowed is 512 bytes per channel, which results in
an event size maximum the same as raw data mode (depending on header format). In zero suppression mode the 
event size depends on the occupancy in the tracker, and the zero suppression thresholds.


%%% Local Variables: 
%%% mode: latex
%%% TeX-master: "DataFormats"
%%% End: 

\graphicspath{{../../../EcalRawData/doc/tex/}}
\section{Silicon Strip Data Format}\label{sec:SiStrip}
The silicon strip tracker Front End Driver readout board will be able to operate in
various modes. The modes enable running for different types of physics events, running
for calibration or commissioning. The modes are [ref????]:
\begin{itemize}
\item Virgin Raw Data mode - in this mode the FED performs no data processing at all. This mode
is intended for commissioning and monitoring running.
\item Processed Raw Data mode - in this mode the FED performs pedestal subtraction and re-ordering (where
the strip data are re-ordered out of APV-MUX order into strip order). This mode is intended for use
when colliding heavy ions.
\item Zero Suppressed mode - in this mode the FED performs pedestal subtraction, re-ordering, common mode noise
calculation and subtraction, and zero suppression. This is intended to be the 'normal' running mode for 
proton-proton collisions.
\item Scope mode - in this mode the FED does not perform any APV header finding, and simply captures the data
entering a channel up to a pre-defined number of bytes. This enables the monitoring of APV tick marks to 
allow for tasks such as syncronization of FED channels.
\end{itemize}

In each mode the FED outputs the data with a DAQ header, a tracker specific header and a DAQ trailer. The DAQ
header was defined in the DAQ TDR [ref????] and consists of event type, level 1 trigger number, bunch crossing number,
source ID, etc. The tracker specific header contains information needed to check tracker and FED operation. 
There are two different tracker specific headers [ref???]: 
\begin{itemize}
\item Full Debug mode - this header contains APV error flags, FED status registers, data lengths, etc. This 
header format willbe used for commissioning, syncronization and raw data mode running.
\item APV Error mode - this will be the 'standard' running mode during a physics run (when the FED is operating
in zero suppressed mode). It is designed to be as small as possible, and so only contains as much information as
is needed for event reconstruction and monitoring.
\end{itemize}

\subsubsection {Silicon Strip FED Event Size}


The event buffer produced by the FED in raw data and processed raw data mode has a fixed size of 49,624 bytes 
(for full debug mode header) or 49,520 bytes (for APV error mode header). In scope mode the event size depends
on the number of bytes captured per FED channel. The maximum allowed is 512 bytes per channel, which results in
an event size maximum the same as raw data mode (depending on header format). In zero suppression mode the 
event size depends on the occupancy in the tracker, and the zero suppression thresholds.


%%% Local Variables: 
%%% mode: latex
%%% TeX-master: "DataFormats"
%%% End: 

\section{Silicon Strip Data Format}\label{sec:SiStrip}
The silicon strip tracker Front End Driver readout board will be able to operate in
various modes. The modes enable running for different types of physics events, running
for calibration or commissioning. The modes are [ref????]:
\begin{itemize}
\item Virgin Raw Data mode - in this mode the FED performs no data processing at all. This mode
is intended for commissioning and monitoring running.
\item Processed Raw Data mode - in this mode the FED performs pedestal subtraction and re-ordering (where
the strip data are re-ordered out of APV-MUX order into strip order). This mode is intended for use
when colliding heavy ions.
\item Zero Suppressed mode - in this mode the FED performs pedestal subtraction, re-ordering, common mode noise
calculation and subtraction, and zero suppression. This is intended to be the 'normal' running mode for 
proton-proton collisions.
\item Scope mode - in this mode the FED does not perform any APV header finding, and simply captures the data
entering a channel up to a pre-defined number of bytes. This enables the monitoring of APV tick marks to 
allow for tasks such as syncronization of FED channels.
\end{itemize}

In each mode the FED outputs the data with a DAQ header, a tracker specific header and a DAQ trailer. The DAQ
header was defined in the DAQ TDR [ref????] and consists of event type, level 1 trigger number, bunch crossing number,
source ID, etc. The tracker specific header contains information needed to check tracker and FED operation. 
There are two different tracker specific headers [ref???]: 
\begin{itemize}
\item Full Debug mode - this header contains APV error flags, FED status registers, data lengths, etc. This 
header format willbe used for commissioning, syncronization and raw data mode running.
\item APV Error mode - this will be the 'standard' running mode during a physics run (when the FED is operating
in zero suppressed mode). It is designed to be as small as possible, and so only contains as much information as
is needed for event reconstruction and monitoring.
\end{itemize}

\subsubsection {Silicon Strip FED Event Size}


The event buffer produced by the FED in raw data and processed raw data mode has a fixed size of 49,624 bytes 
(for full debug mode header) or 49,520 bytes (for APV error mode header). In scope mode the event size depends
on the number of bytes captured per FED channel. The maximum allowed is 512 bytes per channel, which results in
an event size maximum the same as raw data mode (depending on header format). In zero suppression mode the 
event size depends on the occupancy in the tracker, and the zero suppression thresholds.


%%% Local Variables: 
%%% mode: latex
%%% TeX-master: "DataFormats"
%%% End: 

\section{Silicon Strip Data Format}\label{sec:SiStrip}
The silicon strip tracker Front End Driver readout board will be able to operate in
various modes. The modes enable running for different types of physics events, running
for calibration or commissioning. The modes are [ref????]:
\begin{itemize}
\item Virgin Raw Data mode - in this mode the FED performs no data processing at all. This mode
is intended for commissioning and monitoring running.
\item Processed Raw Data mode - in this mode the FED performs pedestal subtraction and re-ordering (where
the strip data are re-ordered out of APV-MUX order into strip order). This mode is intended for use
when colliding heavy ions.
\item Zero Suppressed mode - in this mode the FED performs pedestal subtraction, re-ordering, common mode noise
calculation and subtraction, and zero suppression. This is intended to be the 'normal' running mode for 
proton-proton collisions.
\item Scope mode - in this mode the FED does not perform any APV header finding, and simply captures the data
entering a channel up to a pre-defined number of bytes. This enables the monitoring of APV tick marks to 
allow for tasks such as syncronization of FED channels.
\end{itemize}

In each mode the FED outputs the data with a DAQ header, a tracker specific header and a DAQ trailer. The DAQ
header was defined in the DAQ TDR [ref????] and consists of event type, level 1 trigger number, bunch crossing number,
source ID, etc. The tracker specific header contains information needed to check tracker and FED operation. 
There are two different tracker specific headers [ref???]: 
\begin{itemize}
\item Full Debug mode - this header contains APV error flags, FED status registers, data lengths, etc. This 
header format willbe used for commissioning, syncronization and raw data mode running.
\item APV Error mode - this will be the 'standard' running mode during a physics run (when the FED is operating
in zero suppressed mode). It is designed to be as small as possible, and so only contains as much information as
is needed for event reconstruction and monitoring.
\end{itemize}

\subsubsection {Silicon Strip FED Event Size}


The event buffer produced by the FED in raw data and processed raw data mode has a fixed size of 49,624 bytes 
(for full debug mode header) or 49,520 bytes (for APV error mode header). In scope mode the event size depends
on the number of bytes captured per FED channel. The maximum allowed is 512 bytes per channel, which results in
an event size maximum the same as raw data mode (depending on header format). In zero suppression mode the 
event size depends on the occupancy in the tracker, and the zero suppression thresholds.


%%% Local Variables: 
%%% mode: latex
%%% TeX-master: "DataFormats"
%%% End: 

\section{Silicon Strip Data Format}\label{sec:SiStrip}
The silicon strip tracker Front End Driver readout board will be able to operate in
various modes. The modes enable running for different types of physics events, running
for calibration or commissioning. The modes are [ref????]:
\begin{itemize}
\item Virgin Raw Data mode - in this mode the FED performs no data processing at all. This mode
is intended for commissioning and monitoring running.
\item Processed Raw Data mode - in this mode the FED performs pedestal subtraction and re-ordering (where
the strip data are re-ordered out of APV-MUX order into strip order). This mode is intended for use
when colliding heavy ions.
\item Zero Suppressed mode - in this mode the FED performs pedestal subtraction, re-ordering, common mode noise
calculation and subtraction, and zero suppression. This is intended to be the 'normal' running mode for 
proton-proton collisions.
\item Scope mode - in this mode the FED does not perform any APV header finding, and simply captures the data
entering a channel up to a pre-defined number of bytes. This enables the monitoring of APV tick marks to 
allow for tasks such as syncronization of FED channels.
\end{itemize}

In each mode the FED outputs the data with a DAQ header, a tracker specific header and a DAQ trailer. The DAQ
header was defined in the DAQ TDR [ref????] and consists of event type, level 1 trigger number, bunch crossing number,
source ID, etc. The tracker specific header contains information needed to check tracker and FED operation. 
There are two different tracker specific headers [ref???]: 
\begin{itemize}
\item Full Debug mode - this header contains APV error flags, FED status registers, data lengths, etc. This 
header format willbe used for commissioning, syncronization and raw data mode running.
\item APV Error mode - this will be the 'standard' running mode during a physics run (when the FED is operating
in zero suppressed mode). It is designed to be as small as possible, and so only contains as much information as
is needed for event reconstruction and monitoring.
\end{itemize}

\subsubsection {Silicon Strip FED Event Size}


The event buffer produced by the FED in raw data and processed raw data mode has a fixed size of 49,624 bytes 
(for full debug mode header) or 49,520 bytes (for APV error mode header). In scope mode the event size depends
on the number of bytes captured per FED channel. The maximum allowed is 512 bytes per channel, which results in
an event size maximum the same as raw data mode (depending on header format). In zero suppression mode the 
event size depends on the occupancy in the tracker, and the zero suppression thresholds.


%%% Local Variables: 
%%% mode: latex
%%% TeX-master: "DataFormats"
%%% End: 

\section{Silicon Strip Data Format}\label{sec:SiStrip}
The silicon strip tracker Front End Driver readout board will be able to operate in
various modes. The modes enable running for different types of physics events, running
for calibration or commissioning. The modes are [ref????]:
\begin{itemize}
\item Virgin Raw Data mode - in this mode the FED performs no data processing at all. This mode
is intended for commissioning and monitoring running.
\item Processed Raw Data mode - in this mode the FED performs pedestal subtraction and re-ordering (where
the strip data are re-ordered out of APV-MUX order into strip order). This mode is intended for use
when colliding heavy ions.
\item Zero Suppressed mode - in this mode the FED performs pedestal subtraction, re-ordering, common mode noise
calculation and subtraction, and zero suppression. This is intended to be the 'normal' running mode for 
proton-proton collisions.
\item Scope mode - in this mode the FED does not perform any APV header finding, and simply captures the data
entering a channel up to a pre-defined number of bytes. This enables the monitoring of APV tick marks to 
allow for tasks such as syncronization of FED channels.
\end{itemize}

In each mode the FED outputs the data with a DAQ header, a tracker specific header and a DAQ trailer. The DAQ
header was defined in the DAQ TDR [ref????] and consists of event type, level 1 trigger number, bunch crossing number,
source ID, etc. The tracker specific header contains information needed to check tracker and FED operation. 
There are two different tracker specific headers [ref???]: 
\begin{itemize}
\item Full Debug mode - this header contains APV error flags, FED status registers, data lengths, etc. This 
header format willbe used for commissioning, syncronization and raw data mode running.
\item APV Error mode - this will be the 'standard' running mode during a physics run (when the FED is operating
in zero suppressed mode). It is designed to be as small as possible, and so only contains as much information as
is needed for event reconstruction and monitoring.
\end{itemize}

\subsubsection {Silicon Strip FED Event Size}


The event buffer produced by the FED in raw data and processed raw data mode has a fixed size of 49,624 bytes 
(for full debug mode header) or 49,520 bytes (for APV error mode header). In scope mode the event size depends
on the number of bytes captured per FED channel. The maximum allowed is 512 bytes per channel, which results in
an event size maximum the same as raw data mode (depending on header format). In zero suppression mode the 
event size depends on the occupancy in the tracker, and the zero suppression thresholds.


%%% Local Variables: 
%%% mode: latex
%%% TeX-master: "DataFormats"
%%% End: 

%\section{Conclusions}\label{sec:Conclusions}
...

%%% Local Variables: 
%%% mode: latex
%%% TeX-master: "DataFormats"
%%% End: 

\begin{thebibliography}{99}
\pdfbookmark[1]{References}{References}
%%%%%%%%%%%%%%%%%%%%%%%%%%%%%

%% ECAL References
\bibitem{LECC03} \textbf{Proc. 9th LECC 2003 Workshop, Amsterdam, Holland,
  September 2003}, N. Almeida \textit{et al.}, \textit{Data Concentrator Card and Test System for the CMS ECAL Readout}. 

\bibitem{CALOR04} \textbf{Proc. Int. Conf. Calorimetry in HEP, Perugia,
  Italy, March 2004}, R. Alemany \textit{et al.}, \textit{ECAL Off-Detector Electronics}.

\bibitem{IEEE04} \textbf{Proc. IEEE Nuclear Science Symposium, Rome, October
  2004}, N.Almeida \textit{et al.}, \textit{The Selective Read-Out Processor for the CMS Electromagnetic Calorimeter}.

\bibitem{ELHC00} \textbf{Proc. 6th Workshop on Electronics for LHC
  Experiments, Cracow, Poland, September 2000}, J.Varela, \textit{Timing and Synchronization in the LHC Experiments}.

\bibitem{TRIDAS} The TriDAS Horizontal project Web site: http://cmsdoc.cern.ch/cms/TRIDAS/horizontal.

\bibitem{NOTE_2002/033} \textbf{CMS NOTE 2002/033}, CMS Trigger DAQ group, \textit{CMS L1 Trigger Control System}.

\bibitem{ORCA} The ORCA project Web site: http://cmsdoc.cern.ch/orca.
%% End of ECAL References
%%%%%%%%%%%%%%%%%%%%%%%%%%%%%

\end{thebibliography}


%%% Local Variables: 
%%% mode: latex
%%% TeX-master: "DataFormats"
%%% End: 
 

%This section is just an example of how to format tables with data formats.
\clearpage
\section*{Example of formatting}

Here are some examples of how to create tables with data formats.

Table~\ref{tab:test1} shows an example of a 16-bit table.
% Example of a 16-bit table
\begin{table}[htb]
  \caption{Example of a table for a 16-bit format}\label{tab:test1}
    \begin{bittabular}{16}
      & \bitNumTwoByte
      \bitline{Header 1 & 1 & 0 & 0 & 1 & \field{12}{DMB LA1(11:0)}}
      \bitline{Header 1 & 1 & 0 & 0 & 1 & \field{12}{DMB LA1(23:12)}}
      \bitline{Header 1 & 1 & 0 & 0 & 1 & a & b & \field{5}{CFEB\_ACTIVE} & \field{5}{CFEB\_DAV}}
      \bitline{Header 1 & 1 & 0 & 0 & 1 & \field{12}{DMB BXN}}
      & \bitNumTwoByte
    \end{bittabular}
\end{table}

Table~\ref{tab:test2} shows an example of a 32-bit table.
% Example of a 32-bit table
\begin{table}[htb]
  \caption{Example of a table for a 32-bit format}\label{tab:test2}
    \begin{bittabular}{32}
      & \bitNumFourByte
      \bitline{TDC & 0 & 0 & 0 & 0 & \field{4}{TDC ID} & \field{12}{Event ID} & \field{12}{Bunch ID}}
      \bitline{ROS & 0 & 0 & 0 & \field{5}{ROB ID (0-30)} & \field{12}{Event ID} & \field{12}{Bunch ID}}
      & \bitNumFourByte
    \end{bittabular}
\end{table}

%$\leftarrow$\hfill Width of a line \hfill$\rightarrow$

Table~\ref{tab:test3} shows an example of a 64-bit table in a separate
landscape-oriented page.

% Example of 64-bit table, in a landscape-rotated page
\begin{landscape}
\begin{table}[htb]
  \caption{Example of a table for a 64-bit format}\label{tab:test3}
    \begin{bittabular}{64}
      & \bitNumEightByte
      \bitline{K & 
        \field{4}{BOE\_1} & 
        \field{4}{Evt\_ty} &
        \field{24}{LV1\_id} & 
        \field{12}{BX\_id} & 
        \field{12}{Source\_id} &
        \field{4}{FOV} &
        H & x & \$ & \$}
      \bitline{ D & \field{64}{Sub-detector payload} }
      \bitline{ D & \field{64}{Sub-detector payload} }
      \bitline{K & 
        \field{4}{EOE\_1} & 
        \field{4}{xxxx} &
        \field{24}{Evt\_lgth} & 
        \field{16}{CRC} & 
        \field{4}{xxxx} &
        \field{4}{EVT\_stat} &
        \field{4}{TTS} & T & x & \$ & \$}
      & \bitNumEightByte
    \end{bittabular}
\end{table}

%$\leftarrow$\hfill Width of a line \hfill$\rightarrow$

Legend: 
\begin{itemize}
\item BOE\_n: Identifier for the beginning of an event fragment (BEO\_1 = hex 5)
\item Evt\_ty: Event type identifier (see notes below)
\item LV1\_id: The level-1 event number generated by the TTC system. The first event after a TTC reset is tagged with no. 1
\item BX\_id: The bunch crossing number. Reset on every LHC orbit
\item Source\_id: Unambiguously identify the data source (FED/DCC): 2 bits are reserved for FED internal usage
\item FOV: Version identifier of the FED data format
\item H: when set to '0', the current header word is the last one. When set to '1', another header word is following.
\item EOE\_n: Identifier for the end of an event fragment (EOE\_1 = hex A)
\item Evt\_lgth: The length of the event fragment counted in 64-bit words including header and trailer
\item CRC: Cyclic Redundancy Code of the event fragment including header and trailer (see notes below)
\item Evt\_stat: Event fragment status information (see notes below)
\item TTS: Current values of the TTS bits
\item T: when set to '0', the current trailer word is the last one. When set to '1', another trailer word is following.
\item x: Indicates a reserved bit
\item \$: Indicates a bit used by the S-LINK64 hardware
\end{itemize}

\end{landscape}




%%% Local Variables: 
%%% mode: latex
%%% TeX-master: "DataFormats"
%%% End: 
 


%==============================================================================
\end{document}


%%% Local Variables: 
%%% mode: latex
%%% TeX-master: t
%%% End: 
