%======================================================================
% 
% For instructions see the file README.txt.
%
% Author: N. Amapane  
% $Date: 2005/06/08 18:25:48 $  $Revision: 1.8 $
%
%======================================================================

\documentclass[a4paper]{cmspaper}
\usepackage{lineno} % line numbering for draft
\usepackage{ifthen}
%\usepackage{multido}
\usepackage{array}

% Temporary way to set the input path (until a configuration for
% CMSSW is ready)
\makeatletter 
\def\input@path{{./../../../}} 
\makeatother

%======================================================================
% Check for PDFLaTeX/LaTeX 

\newif\ifpdf
\ifx\pdfoutput\undefined
\pdffalse 
\else
\pdfoutput=1 
\pdftrue
\fi

\ifpdf  % ============== we are running PDFLaTeX
\usepackage{color}
\usepackage[pdftex]{graphicx,graphics}
\usepackage{epsfig}
\usepackage{pdflscape}
\usepackage[bookmarksnumbered,bookmarksopen,bookmarksopenlevel=1,
              colorlinks,
              linkcolor=blue,                      
              citecolor=blue, 
              urlcolor=blue]{hyperref}                                 
\pdfinfo{
  /Title  (Description of Detector Raw Data Formats)
  /Author (...)
  /Keywords (FED Raw data Formats)
  }

\DeclareGraphicsExtensions{.pdf}

\else   % ============== we are not running PDFLaTeX
\usepackage{epsfig}
\usepackage{lscape}

\special{!userdict begin /bop-hook{gsave 200 30 translate
         65 rotate /Times-Roman findfont 216 scalefont setfont
         0 0 moveto 0.93 setgray (DRAFT) show grestore} def end}

\DeclareGraphicsExtensions{.eps}

%fake pdflatex commands.
\newcommand{\pdfbookmark}[3][1]{}
\newcommand{\href}[2]{#2}

\fi

%==============================================================================

% A set of definitions to create tables with data formats.

% The header line with byte numbers
\newcounter{bitCounter}
\def\bitNum{\addtocounter{bitCounter}{-1} \arabic{bitCounter}}
\def\oneByte{\bitNum&\bitNum&\bitNum&\bitNum&\bitNum&\bitNum&\bitNum&\bitNum}

% Row with byte numbers for a 8-bit fragment
\newcommand{\bitNumOneByte}{
  \setcounter{bitCounter}{8}
  \oneByte\\\hline 
}

% Row with byte numbers for a 16-bit fragment
\newcommand{\bitNumTwoByte}{
  \setcounter{bitCounter}{16}
  \oneByte&\oneByte\\\hline 
}

% Row with byte numbers for a 32-bit fragment
\newcommand{\bitNumFourByte}{
  \setcounter{bitCounter}{32}
  \oneByte&\oneByte&\oneByte&\oneByte\\\hline 
}

% Row with byte numbers for a 64-bit fragment
\newcommand{\bitNumEightByte}{
  \setcounter{bitCounter}{64}
  \oneByte&\oneByte&\oneByte&\oneByte&
  \oneByte&\oneByte&\oneByte&\oneByte\\\hline 
}

% A field spanning several bits
\newcommand{\field}[2]{\multicolumn{#1}{c|}{ #2}}

% A row (simple shortcut)
\newcommand{\bitline}[1]{#1\\\hline}

% A tabular with the specified number of bits (plus one leading description field)
\newenvironment{bittabular}[2][3.5mm]
{\hspace{-0.1\linewidth}\begin{minipage}[t]{1.2\linewidth}
\begin{center}
\small\tabcolsep0.2mm
\begin{tabular}{|c|*{#2}{m{#1}<{\centering}|}}\hline}
{\end{tabular}\end{center}\end{minipage}}

%======================================================================

\begin{document}
\begin{titlepage}

  \internalnote{2005/XXX}
  \date{\today}

  \title{Description of Detector Raw Data Formats}
  \begin{Authlist}
    N.~Amapane\Aref{1},
    R.~Alemany Fernandez,
    D.~Kotlinski\Aref{2},
    G.~Bruno,
    M.~Maggi,
    J.~Mans\Aref{3},
    E.~Meschi\Aref{1},

    \Anotfoot{1}{CERN, Geneva, Switzerland}
    \Anotfoot{2}{Paul Scherrer Inst., Villigen, Switzerland}
    \Anotfoot{3}{University of Maryland, USA}

  \end{Authlist}

  \begin{abstract}
    \pdfbookmark[1]{Abstract}{Abstract}
    This note describes the format of the raw data sent to the DAQ by
    the FEDs of the various detectors.
  \end{abstract} 
  
\end{titlepage}

\setcounter{page}{2}

\linenumbers % line numbering for draft

\section{Introduction}\label{sec:Introduction}

This note is intended as a ``working document'' to collect the
description of the format of the raw data transmitted via S-LINK64 by
the Front End Drivers (FED) of the different subdetectors to the DAQ.

In Section~\ref{sec:CommonFormat}, the common format of FED headers
and trailers is described. The detector-specific payload is described
in the subsequent sections.
 

%%% Local Variables: 
%%% mode: latex
%%% TeX-master: "DataFormats"
%%% End: 
 
\section{Common Data Format}\label{sec:CommonFormat}
 
Common fromat; FED headers and trailers.

%When an event fragment is ready to be transmitted to the DAQ, the FED
%encapsulates the data according to the common CMS data format (see
%below) and writes the data into the S-LINK64 port. The encapsulation
%words are control words flagged by the S-LINK64 control bit. Detector
%payload are data words.




%%% Local Variables: 
%%% mode: latex
%%% TeX-master: "DataFormats"
%%% End: 
 
\section{Pixel Data Format}\label{sec:Pixel}
%% Danek Kotlinski, first version from 25/5/05
The pixel Front End Driver (FED) readout board will read 
data from 36 input links, build events, and send the event
packets through the S-link to the DAQ.
During the event building it has to reconstruct pixel
addresses from the 6-level analog signals. This procedure
requires a set of pre-programmed threshold levels. 

Each FED will send to the DAQ a data packet starting with the standard
packet header and ending with the standard packet trailer.
The header includes a bit field identifying uniquely the FED.
Between the header and the trailer there will be a variable number of 
32-bit data words, with one pixel stored per word.
The format of the 32-bit word is:
\begin{itemize}
\item 6-bit Link id, defines the input link to the FED (0-35);
\item 5-bit ROC id, defines the ROC within one link (0-23);
\item 5-bit Double-Column id, define the double column within
on ROC (0-25);
\item 8-bit Pixel id, define the pixel address within on
Double-Column (0-179);
\item 8-bit ADC value, the signal amplitude, extracted from a
10-bit ADC.
\end{itemize}
Table~\ref{tab:pix1} summarizes the pixel format.

Note that in our data format the source id (FED number) is not 
included. We depend on the source id included in the DAQ S-link 
header. This header should be available in the code which will 
transform the raw data format to the format used in the reconstruction 
code.

% Pixel data format
\begin{table}[htb]
  \caption{Pixel readout format}\label{tab:pix1}
    \begin{bittabular}{32}
      & \bitNumFourByte
      \bitline{pixel&\field{6}{LINK-ID}&\field{5}{ROC-ID}&\field{5}{DCOL-ID}&
	\field{8}{PIX-ID} & \field{8}{ADC}}
      \bitline{ & 5&4&3&2&1&0 & 4&3&2&1&0 & 4&3&2&1&0 & 7&6&5&4&3&2&1&0 &
	7&6&5&4&3&2&1&0}
     %%  & \bitNumFourByte
    \end{bittabular}
\end{table}

Based on this format the FED data volume in bytes is calculated as
4 * (3 + number-of-pixels).
The pixel barrel readout has been arranged in such a way to
approximately fit the requirement of 2.0~kB per FED at high luminosity.
For the low luminosity pp collisions, the pixel data volume for
the barrel FEDs will be around 0.6~kB per event.
The difference between various FEDs should be around 10\%.
For the pixel endcaps the data volume per FED will be smaller, 
about 1.8(0.55) kB per event at high(low) luminosity.

When reading data over VME an alternative "raw" data format
is foreseen in addition to the standard one. This format
will be only used during the level calibration procedure
performed in a XDAQ client application and will never be used
in ORCA.

%%% Local Variables: 
%%% mode: latex
%%% TeX-master: "DataFormats"
%%% End: 

\section{Pixel Data Format}\label{sec:Pixel}
%% Danek Kotlinski, first version from 25/5/05
The pixel Front End Driver (FED) readout board will read 
data from 36 input links, build events, and send the event
packets through the S-link to the DAQ.
During the event building it has to reconstruct pixel
addresses from the 6-level analog signals. This procedure
requires a set of pre-programmed threshold levels. 

Each FED will send to the DAQ a data packet starting with the standard
packet header and ending with the standard packet trailer.
The header includes a bit field identifying uniquely the FED.
Between the header and the trailer there will be a variable number of 
32-bit data words, with one pixel stored per word.
The format of the 32-bit word is:
\begin{itemize}
\item 6-bit Link id, defines the input link to the FED (0-35);
\item 5-bit ROC id, defines the ROC within one link (0-23);
\item 5-bit Double-Column id, define the double column within
on ROC (0-25);
\item 8-bit Pixel id, define the pixel address within on
Double-Column (0-179);
\item 8-bit ADC value, the signal amplitude, extracted from a
10-bit ADC.
\end{itemize}
Table~\ref{tab:pix1} summarizes the pixel format.

Note that in our data format the source id (FED number) is not 
included. We depend on the source id included in the DAQ S-link 
header. This header should be available in the code which will 
transform the raw data format to the format used in the reconstruction 
code.

% Pixel data format
\begin{table}[htb]
  \caption{Pixel readout format}\label{tab:pix1}
    \begin{bittabular}{32}
      & \bitNumFourByte
      \bitline{pixel&\field{6}{LINK-ID}&\field{5}{ROC-ID}&\field{5}{DCOL-ID}&
	\field{8}{PIX-ID} & \field{8}{ADC}}
      \bitline{ & 5&4&3&2&1&0 & 4&3&2&1&0 & 4&3&2&1&0 & 7&6&5&4&3&2&1&0 &
	7&6&5&4&3&2&1&0}
     %%  & \bitNumFourByte
    \end{bittabular}
\end{table}

Based on this format the FED data volume in bytes is calculated as
4 * (3 + number-of-pixels).
The pixel barrel readout has been arranged in such a way to
approximately fit the requirement of 2.0~kB per FED at high luminosity.
For the low luminosity pp collisions, the pixel data volume for
the barrel FEDs will be around 0.6~kB per event.
The difference between various FEDs should be around 10\%.
For the pixel endcaps the data volume per FED will be smaller, 
about 1.8(0.55) kB per event at high(low) luminosity.

When reading data over VME an alternative "raw" data format
is foreseen in addition to the standard one. This format
will be only used during the level calibration procedure
performed in a XDAQ client application and will never be used
in ORCA.

%%% Local Variables: 
%%% mode: latex
%%% TeX-master: "DataFormats"
%%% End: 

\graphicspath{{../../../EcalRawData/doc/tex/}}
\section{Pixel Data Format}\label{sec:Pixel}
%% Danek Kotlinski, first version from 25/5/05
The pixel Front End Driver (FED) readout board will read 
data from 36 input links, build events, and send the event
packets through the S-link to the DAQ.
During the event building it has to reconstruct pixel
addresses from the 6-level analog signals. This procedure
requires a set of pre-programmed threshold levels. 

Each FED will send to the DAQ a data packet starting with the standard
packet header and ending with the standard packet trailer.
The header includes a bit field identifying uniquely the FED.
Between the header and the trailer there will be a variable number of 
32-bit data words, with one pixel stored per word.
The format of the 32-bit word is:
\begin{itemize}
\item 6-bit Link id, defines the input link to the FED (0-35);
\item 5-bit ROC id, defines the ROC within one link (0-23);
\item 5-bit Double-Column id, define the double column within
on ROC (0-25);
\item 8-bit Pixel id, define the pixel address within on
Double-Column (0-179);
\item 8-bit ADC value, the signal amplitude, extracted from a
10-bit ADC.
\end{itemize}
Table~\ref{tab:pix1} summarizes the pixel format.

Note that in our data format the source id (FED number) is not 
included. We depend on the source id included in the DAQ S-link 
header. This header should be available in the code which will 
transform the raw data format to the format used in the reconstruction 
code.

% Pixel data format
\begin{table}[htb]
  \caption{Pixel readout format}\label{tab:pix1}
    \begin{bittabular}{32}
      & \bitNumFourByte
      \bitline{pixel&\field{6}{LINK-ID}&\field{5}{ROC-ID}&\field{5}{DCOL-ID}&
	\field{8}{PIX-ID} & \field{8}{ADC}}
      \bitline{ & 5&4&3&2&1&0 & 4&3&2&1&0 & 4&3&2&1&0 & 7&6&5&4&3&2&1&0 &
	7&6&5&4&3&2&1&0}
     %%  & \bitNumFourByte
    \end{bittabular}
\end{table}

Based on this format the FED data volume in bytes is calculated as
4 * (3 + number-of-pixels).
The pixel barrel readout has been arranged in such a way to
approximately fit the requirement of 2.0~kB per FED at high luminosity.
For the low luminosity pp collisions, the pixel data volume for
the barrel FEDs will be around 0.6~kB per event.
The difference between various FEDs should be around 10\%.
For the pixel endcaps the data volume per FED will be smaller, 
about 1.8(0.55) kB per event at high(low) luminosity.

When reading data over VME an alternative "raw" data format
is foreseen in addition to the standard one. This format
will be only used during the level calibration procedure
performed in a XDAQ client application and will never be used
in ORCA.

%%% Local Variables: 
%%% mode: latex
%%% TeX-master: "DataFormats"
%%% End: 

\section{Pixel Data Format}\label{sec:Pixel}
%% Danek Kotlinski, first version from 25/5/05
The pixel Front End Driver (FED) readout board will read 
data from 36 input links, build events, and send the event
packets through the S-link to the DAQ.
During the event building it has to reconstruct pixel
addresses from the 6-level analog signals. This procedure
requires a set of pre-programmed threshold levels. 

Each FED will send to the DAQ a data packet starting with the standard
packet header and ending with the standard packet trailer.
The header includes a bit field identifying uniquely the FED.
Between the header and the trailer there will be a variable number of 
32-bit data words, with one pixel stored per word.
The format of the 32-bit word is:
\begin{itemize}
\item 6-bit Link id, defines the input link to the FED (0-35);
\item 5-bit ROC id, defines the ROC within one link (0-23);
\item 5-bit Double-Column id, define the double column within
on ROC (0-25);
\item 8-bit Pixel id, define the pixel address within on
Double-Column (0-179);
\item 8-bit ADC value, the signal amplitude, extracted from a
10-bit ADC.
\end{itemize}
Table~\ref{tab:pix1} summarizes the pixel format.

Note that in our data format the source id (FED number) is not 
included. We depend on the source id included in the DAQ S-link 
header. This header should be available in the code which will 
transform the raw data format to the format used in the reconstruction 
code.

% Pixel data format
\begin{table}[htb]
  \caption{Pixel readout format}\label{tab:pix1}
    \begin{bittabular}{32}
      & \bitNumFourByte
      \bitline{pixel&\field{6}{LINK-ID}&\field{5}{ROC-ID}&\field{5}{DCOL-ID}&
	\field{8}{PIX-ID} & \field{8}{ADC}}
      \bitline{ & 5&4&3&2&1&0 & 4&3&2&1&0 & 4&3&2&1&0 & 7&6&5&4&3&2&1&0 &
	7&6&5&4&3&2&1&0}
     %%  & \bitNumFourByte
    \end{bittabular}
\end{table}

Based on this format the FED data volume in bytes is calculated as
4 * (3 + number-of-pixels).
The pixel barrel readout has been arranged in such a way to
approximately fit the requirement of 2.0~kB per FED at high luminosity.
For the low luminosity pp collisions, the pixel data volume for
the barrel FEDs will be around 0.6~kB per event.
The difference between various FEDs should be around 10\%.
For the pixel endcaps the data volume per FED will be smaller, 
about 1.8(0.55) kB per event at high(low) luminosity.

When reading data over VME an alternative "raw" data format
is foreseen in addition to the standard one. This format
will be only used during the level calibration procedure
performed in a XDAQ client application and will never be used
in ORCA.

%%% Local Variables: 
%%% mode: latex
%%% TeX-master: "DataFormats"
%%% End: 

\section{Pixel Data Format}\label{sec:Pixel}
%% Danek Kotlinski, first version from 25/5/05
The pixel Front End Driver (FED) readout board will read 
data from 36 input links, build events, and send the event
packets through the S-link to the DAQ.
During the event building it has to reconstruct pixel
addresses from the 6-level analog signals. This procedure
requires a set of pre-programmed threshold levels. 

Each FED will send to the DAQ a data packet starting with the standard
packet header and ending with the standard packet trailer.
The header includes a bit field identifying uniquely the FED.
Between the header and the trailer there will be a variable number of 
32-bit data words, with one pixel stored per word.
The format of the 32-bit word is:
\begin{itemize}
\item 6-bit Link id, defines the input link to the FED (0-35);
\item 5-bit ROC id, defines the ROC within one link (0-23);
\item 5-bit Double-Column id, define the double column within
on ROC (0-25);
\item 8-bit Pixel id, define the pixel address within on
Double-Column (0-179);
\item 8-bit ADC value, the signal amplitude, extracted from a
10-bit ADC.
\end{itemize}
Table~\ref{tab:pix1} summarizes the pixel format.

Note that in our data format the source id (FED number) is not 
included. We depend on the source id included in the DAQ S-link 
header. This header should be available in the code which will 
transform the raw data format to the format used in the reconstruction 
code.

% Pixel data format
\begin{table}[htb]
  \caption{Pixel readout format}\label{tab:pix1}
    \begin{bittabular}{32}
      & \bitNumFourByte
      \bitline{pixel&\field{6}{LINK-ID}&\field{5}{ROC-ID}&\field{5}{DCOL-ID}&
	\field{8}{PIX-ID} & \field{8}{ADC}}
      \bitline{ & 5&4&3&2&1&0 & 4&3&2&1&0 & 4&3&2&1&0 & 7&6&5&4&3&2&1&0 &
	7&6&5&4&3&2&1&0}
     %%  & \bitNumFourByte
    \end{bittabular}
\end{table}

Based on this format the FED data volume in bytes is calculated as
4 * (3 + number-of-pixels).
The pixel barrel readout has been arranged in such a way to
approximately fit the requirement of 2.0~kB per FED at high luminosity.
For the low luminosity pp collisions, the pixel data volume for
the barrel FEDs will be around 0.6~kB per event.
The difference between various FEDs should be around 10\%.
For the pixel endcaps the data volume per FED will be smaller, 
about 1.8(0.55) kB per event at high(low) luminosity.

When reading data over VME an alternative "raw" data format
is foreseen in addition to the standard one. This format
will be only used during the level calibration procedure
performed in a XDAQ client application and will never be used
in ORCA.

%%% Local Variables: 
%%% mode: latex
%%% TeX-master: "DataFormats"
%%% End: 

\section{Pixel Data Format}\label{sec:Pixel}
%% Danek Kotlinski, first version from 25/5/05
The pixel Front End Driver (FED) readout board will read 
data from 36 input links, build events, and send the event
packets through the S-link to the DAQ.
During the event building it has to reconstruct pixel
addresses from the 6-level analog signals. This procedure
requires a set of pre-programmed threshold levels. 

Each FED will send to the DAQ a data packet starting with the standard
packet header and ending with the standard packet trailer.
The header includes a bit field identifying uniquely the FED.
Between the header and the trailer there will be a variable number of 
32-bit data words, with one pixel stored per word.
The format of the 32-bit word is:
\begin{itemize}
\item 6-bit Link id, defines the input link to the FED (0-35);
\item 5-bit ROC id, defines the ROC within one link (0-23);
\item 5-bit Double-Column id, define the double column within
on ROC (0-25);
\item 8-bit Pixel id, define the pixel address within on
Double-Column (0-179);
\item 8-bit ADC value, the signal amplitude, extracted from a
10-bit ADC.
\end{itemize}
Table~\ref{tab:pix1} summarizes the pixel format.

Note that in our data format the source id (FED number) is not 
included. We depend on the source id included in the DAQ S-link 
header. This header should be available in the code which will 
transform the raw data format to the format used in the reconstruction 
code.

% Pixel data format
\begin{table}[htb]
  \caption{Pixel readout format}\label{tab:pix1}
    \begin{bittabular}{32}
      & \bitNumFourByte
      \bitline{pixel&\field{6}{LINK-ID}&\field{5}{ROC-ID}&\field{5}{DCOL-ID}&
	\field{8}{PIX-ID} & \field{8}{ADC}}
      \bitline{ & 5&4&3&2&1&0 & 4&3&2&1&0 & 4&3&2&1&0 & 7&6&5&4&3&2&1&0 &
	7&6&5&4&3&2&1&0}
     %%  & \bitNumFourByte
    \end{bittabular}
\end{table}

Based on this format the FED data volume in bytes is calculated as
4 * (3 + number-of-pixels).
The pixel barrel readout has been arranged in such a way to
approximately fit the requirement of 2.0~kB per FED at high luminosity.
For the low luminosity pp collisions, the pixel data volume for
the barrel FEDs will be around 0.6~kB per event.
The difference between various FEDs should be around 10\%.
For the pixel endcaps the data volume per FED will be smaller, 
about 1.8(0.55) kB per event at high(low) luminosity.

When reading data over VME an alternative "raw" data format
is foreseen in addition to the standard one. This format
will be only used during the level calibration procedure
performed in a XDAQ client application and will never be used
in ORCA.

%%% Local Variables: 
%%% mode: latex
%%% TeX-master: "DataFormats"
%%% End: 

\section{Pixel Data Format}\label{sec:Pixel}
%% Danek Kotlinski, first version from 25/5/05
The pixel Front End Driver (FED) readout board will read 
data from 36 input links, build events, and send the event
packets through the S-link to the DAQ.
During the event building it has to reconstruct pixel
addresses from the 6-level analog signals. This procedure
requires a set of pre-programmed threshold levels. 

Each FED will send to the DAQ a data packet starting with the standard
packet header and ending with the standard packet trailer.
The header includes a bit field identifying uniquely the FED.
Between the header and the trailer there will be a variable number of 
32-bit data words, with one pixel stored per word.
The format of the 32-bit word is:
\begin{itemize}
\item 6-bit Link id, defines the input link to the FED (0-35);
\item 5-bit ROC id, defines the ROC within one link (0-23);
\item 5-bit Double-Column id, define the double column within
on ROC (0-25);
\item 8-bit Pixel id, define the pixel address within on
Double-Column (0-179);
\item 8-bit ADC value, the signal amplitude, extracted from a
10-bit ADC.
\end{itemize}
Table~\ref{tab:pix1} summarizes the pixel format.

Note that in our data format the source id (FED number) is not 
included. We depend on the source id included in the DAQ S-link 
header. This header should be available in the code which will 
transform the raw data format to the format used in the reconstruction 
code.

% Pixel data format
\begin{table}[htb]
  \caption{Pixel readout format}\label{tab:pix1}
    \begin{bittabular}{32}
      & \bitNumFourByte
      \bitline{pixel&\field{6}{LINK-ID}&\field{5}{ROC-ID}&\field{5}{DCOL-ID}&
	\field{8}{PIX-ID} & \field{8}{ADC}}
      \bitline{ & 5&4&3&2&1&0 & 4&3&2&1&0 & 4&3&2&1&0 & 7&6&5&4&3&2&1&0 &
	7&6&5&4&3&2&1&0}
     %%  & \bitNumFourByte
    \end{bittabular}
\end{table}

Based on this format the FED data volume in bytes is calculated as
4 * (3 + number-of-pixels).
The pixel barrel readout has been arranged in such a way to
approximately fit the requirement of 2.0~kB per FED at high luminosity.
For the low luminosity pp collisions, the pixel data volume for
the barrel FEDs will be around 0.6~kB per event.
The difference between various FEDs should be around 10\%.
For the pixel endcaps the data volume per FED will be smaller, 
about 1.8(0.55) kB per event at high(low) luminosity.

When reading data over VME an alternative "raw" data format
is foreseen in addition to the standard one. This format
will be only used during the level calibration procedure
performed in a XDAQ client application and will never be used
in ORCA.

%%% Local Variables: 
%%% mode: latex
%%% TeX-master: "DataFormats"
%%% End: 

\section{Conclusions}\label{sec:Conclusions}
...

%%% Local Variables: 
%%% mode: latex
%%% TeX-master: "DataFormats"
%%% End: 

\begin{thebibliography}{99}
\pdfbookmark[1]{References}{References}
%%%%%%%%%%%%%%%%%%%%%%%%%%%%%

%% ECAL References
\bibitem{LECC03} \textbf{Proc. 9th LECC 2003 Workshop, Amsterdam, Holland,
  September 2003}, N. Almeida \textit{et al.}, \textit{Data Concentrator Card and Test System for the CMS ECAL Readout}. 

\bibitem{CALOR04} \textbf{Proc. Int. Conf. Calorimetry in HEP, Perugia,
  Italy, March 2004}, R. Alemany \textit{et al.}, \textit{ECAL Off-Detector Electronics}.

\bibitem{IEEE04} \textbf{Proc. IEEE Nuclear Science Symposium, Rome, October
  2004}, N.Almeida \textit{et al.}, \textit{The Selective Read-Out Processor for the CMS Electromagnetic Calorimeter}.

\bibitem{ELHC00} \textbf{Proc. 6th Workshop on Electronics for LHC
  Experiments, Cracow, Poland, September 2000}, J.Varela, \textit{Timing and Synchronization in the LHC Experiments}.

\bibitem{TRIDAS} The TriDAS Horizontal project Web site: http://cmsdoc.cern.ch/cms/TRIDAS/horizontal.

\bibitem{NOTE_2002/033} \textbf{CMS NOTE 2002/033}, CMS Trigger DAQ group, \textit{CMS L1 Trigger Control System}.

\bibitem{ORCA} The ORCA project Web site: http://cmsdoc.cern.ch/orca.
%% End of ECAL References
%%%%%%%%%%%%%%%%%%%%%%%%%%%%%

\end{thebibliography}


%%% Local Variables: 
%%% mode: latex
%%% TeX-master: "DataFormats"
%%% End: 
 

%This section is just an example of how to format tables with data formats.
\clearpage
\section*{Example of formatting}

Here are some examples of how to create tables with data formats.

Table~\ref{tab:test1} shows an example of a 16-bit table.
% Example of a 16-bit table
\begin{table}[htb]
  \caption{Example of a table for a 16-bit format}\label{tab:test1}
    \begin{bittabular}{16}
      & \bitNumTwoByte
      \bitline{Header 1 & 1 & 0 & 0 & 1 & \field{12}{DMB LA1(11:0)}}
      \bitline{Header 1 & 1 & 0 & 0 & 1 & \field{12}{DMB LA1(23:12)}}
      \bitline{Header 1 & 1 & 0 & 0 & 1 & a & b & \field{5}{CFEB\_ACTIVE} & \field{5}{CFEB\_DAV}}
      \bitline{Header 1 & 1 & 0 & 0 & 1 & \field{12}{DMB BXN}}
      & \bitNumTwoByte
    \end{bittabular}
\end{table}

Table~\ref{tab:test2} shows an example of a 32-bit table.
% Example of a 32-bit table
\begin{table}[htb]
  \caption{Example of a table for a 32-bit format}\label{tab:test2}
    \begin{bittabular}{32}
      & \bitNumFourByte
      \bitline{TDC & 0 & 0 & 0 & 0 & \field{4}{TDC ID} & \field{12}{Event ID} & \field{12}{Bunch ID}}
      \bitline{ROS & 0 & 0 & 0 & \field{5}{ROB ID (0-30)} & \field{12}{Event ID} & \field{12}{Bunch ID}}
      & \bitNumFourByte
    \end{bittabular}
\end{table}

%$\leftarrow$\hfill Width of a line \hfill$\rightarrow$

Table~\ref{tab:test3} shows an example of a 64-bit table in a separate
landscape-oriented page.

% Example of 64-bit table, in a landscape-rotated page
\begin{landscape}
\begin{table}[htb]
  \caption{Example of a table for a 64-bit format}\label{tab:test3}
    \begin{bittabular}{64}
      & \bitNumEightByte
      \bitline{K & 
        \field{4}{BOE\_1} & 
        \field{4}{Evt\_ty} &
        \field{24}{LV1\_id} & 
        \field{12}{BX\_id} & 
        \field{12}{Source\_id} &
        \field{4}{FOV} &
        H & x & \$ & \$}
      \bitline{ D & \field{64}{Sub-detector payload} }
      \bitline{ D & \field{64}{Sub-detector payload} }
      \bitline{K & 
        \field{4}{EOE\_1} & 
        \field{4}{xxxx} &
        \field{24}{Evt\_lgth} & 
        \field{16}{CRC} & 
        \field{4}{xxxx} &
        \field{4}{EVT\_stat} &
        \field{4}{TTS} & T & x & \$ & \$}
      & \bitNumEightByte
    \end{bittabular}
\end{table}

%$\leftarrow$\hfill Width of a line \hfill$\rightarrow$

Legend: 
\begin{itemize}
\item BOE\_n: Identifier for the beginning of an event fragment (BEO\_1 = hex 5)
\item Evt\_ty: Event type identifier (see notes below)
\item LV1\_id: The level-1 event number generated by the TTC system. The first event after a TTC reset is tagged with no. 1
\item BX\_id: The bunch crossing number. Reset on every LHC orbit
\item Source\_id: Unambiguously identify the data source (FED/DCC): 2 bits are reserved for FED internal usage
\item FOV: Version identifier of the FED data format
\item H: when set to '0', the current header word is the last one. When set to '1', another header word is following.
\item EOE\_n: Identifier for the end of an event fragment (EOE\_1 = hex A)
\item Evt\_lgth: The length of the event fragment counted in 64-bit words including header and trailer
\item CRC: Cyclic Redundancy Code of the event fragment including header and trailer (see notes below)
\item Evt\_stat: Event fragment status information (see notes below)
\item TTS: Current values of the TTS bits
\item T: when set to '0', the current trailer word is the last one. When set to '1', another trailer word is following.
\item x: Indicates a reserved bit
\item \$: Indicates a bit used by the S-LINK64 hardware
\end{itemize}

\end{landscape}




%%% Local Variables: 
%%% mode: latex
%%% TeX-master: "DataFormats"
%%% End: 
 


%==============================================================================
\end{document}


%%% Local Variables: 
%%% mode: latex
%%% TeX-master: t
%%% End: 
