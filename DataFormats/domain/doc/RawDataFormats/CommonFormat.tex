\label{sec:CommonFormat}
\section[Common Data Format]{Common Data Format\footnote{Section taken from
http://cmsdoc.cern.ch/cms/TRIDAS/horizontal/RUWG/DAQ\_IF\_guide/DAQ\_IF\_guide.html\#CDF}}

When an event fragment is ready to be transmitted to the DAQ, the FED
encapsulates the data according to the common CMS data format of
Table~\ref{tab:commonformat} and writes the data into the S-LINK64 port. The
encapsulation words are control words flagged by the S-LINK64 control
bit. Detector payload are data words.

The DAQ will neither look into the sub-detector payload nor touch the
encapsulation: the header and the trailer will be transmitted "as is"
to the filter units without any modification.

The fields in the header and trailer words of
Table~\ref{tab:commonformat} have the following meaning:
\begin{description}
\item {\bf DAQ Header}
\begin{description}
\item {\bf BOE\_n}: Identifier for the beginning of an event fragment (BOE\_1 = '010').
\item {\bf Evt\_ty}: Event Trigger type identifier, defined by the central
  DAQ, cf~ Table~\ref{tab:TRIGGER_TYPE}~\cite{NOTE_2002/033}.
\item {\bf LV1\_id}: The level-1 event number generated by the TTC
  system. The first event after a TTC reset (B-Go5 Resync, B-Go6
  HardReset, or B-Go7 ResetEventCounter) is tagged with Event Number =
  1. The Event Counter has 24 bits, the event following Event Number =
  $2^{24}-1$ is Event Number = 0.
\item {\bf BX\_id}: The bunch crossing number, generated by the TTC
  system. Reset on every LHC orbit.
\item {\bf Source\_id}: Identifier of the FED; 2 bits are reserved for FED internal usage.
\item {\bf FOV}: Version identifier of the FED data format.
\item {\bf H}: When set to '0', the current header word is the last
  one. When set to '1', another header word can follow\footnote{in the
    ECAL DCC, H is set to ´1´.}.%FIXME: NA, taken from the ECAL part
\item {\bf \$}: Bit used by the S-LINK64 hardware
\end{description}
\item {\bf DAQ Trailer}
\begin{description}
\item {\bf EOE\_n}: Identifier for the end of an event fragment (EOE\_1 = '1010').
\item {\bf Evt\_lgth}: The length of the event fragment counted in 64-bit words including header and trailer.
\item {\bf CRC}: Cyclic Redundancy Code of the event fragment including
  header and trailer. %FIXME: see notes on Attila's web page
\item {\bf Evt\_stat}: Event fragment status information (defined by the central DAQ).
\item {\bf TTS}: Current value of the Trigger Throttling System bits.
\item {\bf T}: When set to '0', the current trailer word is the last one. When set to '1', another trailer words can follow\footnote{in the
    ECAL DCC, T is set to ´0´.}.%FIXME: NA, taken from the ECAL part.
\item x: Indicates a reserved bit.
\item {\bf \$}: Bit used by the S-LINK64 hardware.
\end{description}
\end{description}

% 64-bit table, in a landscape-rotated page
\begin{landscape}
\thispagestyle{empty}
\begin{table}[htbp]
  \caption{Common FED data format.}\label{tab:commonformat}
    \begin{bittabular}{64}
      & \bitNumEightByte
      \bitline{K & 
        \field{4}{BOE\_1} & 
        \field{4}{Evt\_ty} &
        \field{24}{LV1\_id} & 
        \field{12}{BX\_id} & 
        \field{12}{Source\_id} &
        \field{4}{FOV} &
        H & x & \$ & \$}
      \bitline{ D & \field{64}{Sub-detector payload} }
      \bitline{ D & \field{64}{Sub-detector payload} }
      \vdotline{64}
      \bitline{K & 
        \field{4}{EOE\_1} & 
        \field{4}{xxxx} &
        \field{24}{Evt\_lgth} & 
        \field{16}{CRC} & 
        \field{4}{xxxx} &
        \field{4}{EVT\_stat} &
        \field{4}{TTS} & T & x & \$ & \$}
      & \bitNumEightByte
    \end{bittabular}
\end{table}

\vspace{1cm}

%$\leftarrow$\hfill Width of a line \hfill$\rightarrow$
\begin{table}[htbp]
\begin{center}
\caption{The TRIGGER TYPE description.}
\label{tab:TRIGGER_TYPE}
\begin{tabular}{|c|l|}   \hline 
{\bf TRIGGER TYPE} & {\bf Description} \\ \hline
0001 &  Physics trigger \\ \hline
0010 &  Calibration trigger \\ \hline
0011 &  Test trigger \\ \hline
0100 &  Technical trigger (external trigger) \\ \hline
0101 &  Simulated events (reserved for DAQ usage) \\ \hline
0110 &  Traced events (reserved for DAQ usage) \\ \hline
1111 &  Error \\ \hline
\end{tabular}
\end{center}
\end{table}


\end{landscape}




%%% Local Variables: 
%%% mode: latex
%%% TeX-master: "DataFormats"
%%% End: 
