\section{Silicon Strip Data Format}\label{sec:SiStrip}
The silicon strip tracker Front End Driver readout board will be able to operate in
various modes. The modes enable running for different types of physics events, running
for calibration or commissioning. The modes are [ref????]:
\begin{itemize}
\item Virgin Raw Data mode - in this mode the FED performs no data processing at all. This mode
is intended for commissioning and monitoring running.
\item Processed Raw Data mode - in this mode the FED performs pedestal subtraction and re-ordering (where
the strip data are re-ordered out of APV-MUX order into strip order). This mode is intended for use
when colliding heavy ions.
\item Zero Suppressed mode - in this mode the FED performs pedestal subtraction, re-ordering, common mode noise
calculation and subtraction, and zero suppression. This is intended to be the 'normal' running mode for 
proton-proton collisions.
\item Scope mode - in this mode the FED does not perform any APV header finding, and simply captures the data
entering a channel up to a pre-defined number of bytes. This enables the monitoring of APV tick marks to 
allow for tasks such as syncronization of FED channels.
\end{itemize}

In each mode the FED outputs the data with a DAQ header, a tracker specific header and a DAQ trailer. The DAQ
header was defined in the DAQ TDR [ref????] and consists of event type, level 1 trigger number, bunch crossing number,
source ID, etc. The tracker specific header contains information needed to check tracker and FED operation. 
There are two different tracker specific headers [ref???]: 
\begin{itemize}
\item Full Debug mode - this header contains APV error flags, FED status registers, data lengths, etc. This 
header format willbe used for commissioning, syncronization and raw data mode running.
\item APV Error mode - this will be the 'standard' running mode during a physics run (when the FED is operating
in zero suppressed mode). It is designed to be as small as possible, and so only contains as much information as
is needed for event reconstruction and monitoring.
\end{itemize}

\subsubsection {Silicon Strip FED Event Size}


The event buffer produced by the FED in raw data and processed raw data mode has a fixed size of 49,624 bytes 
(for full debug mode header) or 49,520 bytes (for APV error mode header). In scope mode the event size depends
on the number of bytes captured per FED channel. The maximum allowed is 512 bytes per channel, which results in
an event size maximum the same as raw data mode (depending on header format). In zero suppression mode the 
event size depends on the occupancy in the tracker, and the zero suppression thresholds.


%%% Local Variables: 
%%% mode: latex
%%% TeX-master: "DataFormats"
%%% End: 
